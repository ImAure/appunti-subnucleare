\chapter{Prefazione}
    Per gli ultimi sette anni---o giù di lì---il prof. Sebastiano Albergo ha tenuto il modulo di \textit{Istituzioni di Fisica Subnucleare} del corso di \textit{Istituzioni di Fisica Nucleare e Subnucleare} presso il Dipartimento di Fisica e Astronomia ``\textit{Ettore Majorana}'' dell'Università di Catania. Durante l'Anno Accademico 2024--2025 ha deciso di prendere un anno sabbatico ed è quindi stato sostituito dalla prof.ssa Rossella Caruso che ha cercato di mantenere al meglio la continuità del corso.\footnote{Continuità rispetto al programma del prof. Albergo e non rispetto alle lezioni, che si sono tenute con la stessa continuità di cui gode la funzione di Thomae sui numeri razionali.}

    C'era momento migliore per scrivere degli appunti in \LaTeX? Probabilmente sì. C'era momento in cui era più necessario averli? Probabilmente no. Quindi eccoci qua!

    Ovviamente siamo studenti, e gli studenti non sono pagati per scrivere appunti. Ciononostante la necessità di un materiale organico e completo rispetto agli argomenti trattati in aula---che non sia necessariamente un libro di testo di novecento pagine per un corso di 3 CFU---ci ha spinti a fare del nostro meglio.

    Sia $n \in \bbN$ un numero naturale arbitrariamente grande. In questi appunti ci saranno almeno $n$ errori e $n^2$ cose non chiare. Per questo invitiamo chiunque legga questi appunti a segnalare problemi e suggerire miglioramenti. È facilissimo, basta aprire una issue su GitHub.

    {\flushright\theauthor}
 
