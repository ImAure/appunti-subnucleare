\chapter{La Fisica del Neutrino}
    \section{Il problema del decadimento \texorpdfstring{\betap}{beta}}
        Il decadimento \betap\ è un fenomeno nucleare che avviene quando un nucleo instabile si trasforma in uno più stabile trasformando un protone in neutrone o viceversa; questa trasformazione viene tipicamente osservata attraverso l'emissione di una particella $\betap^\pm$, ovvero un elettrone ($\betap^-$) o un positrone ($\betap^+$).

        Il problema così impostato tuttavia dovrebbe portare all'osservazione di particelle \betap\ monoenergetiche, in particolare con un'energia cinetica pari al difetto di massa del nucleo. Nel caso del decadimento $\betap^-$ il comportamento previsto applicando il modello a due corpi è
        \begin{equation*}
            \ce{$\pqty{A, Z}$ -> $\pqty{A, Z+1}$ + e^-}
            \mycomma
        \end{equation*}
        dove $A$ è il numero di massa e $Z$ il numero atomico del nucleo. Il difetto di massa è dato dalla differenza tra la massa del nucleo iniziale e quella del nucleo finale più quella dell'elettrone: 
        \begin{equation*}
            Q_\betap = m\pqty{A, Z} - m\pqty{A, Z+1} - m\pqty*{\txt{e}^-}
            \mycomma
        \end{equation*}
        essendo $Q_\betap$ è il \Qv\ del decadimento. 
        
        Il problema è che lo spettro energetico dei \betap---osservato in un importante esperimento da Ellis e Wooster---risulta continuo, con un massimo pari al \Qv\ del decadimento, e non piccato come previsto da un decadimento monoenergetico. Per spiegare questa discrepanza, W. Pauli nel 1903 propose con una lettera l'introduzione di una nuova particella da chiamare \emph{neutrone}\footnote{Il neutrone vero e proprio sarebbe stato scoperto due anni dopo.} con massa e proprietà simili all'elettrone ma di carica neutra, che venisse emesso insieme all'elettrone per bilanciare il decadimento e preservare le leggi di conservazione.

        Dopo la scoperta del neutrone in quanto nucleone, molto più massiccio di un elettrone e che non veniva osservato come emissione nei decadimenti \betap, E. Fermi propose nel 1932--1933 di cambiare il nome della particella ipotetica in \emph{neutrino}, essendo questo molto più piccolo e sfuggente di un neutrone.

        \subsection{La proposta di Fermi per il decadimento \texorpdfstring{\betap}{beta}}
            Una delle grandi intuizioni di Fermi fu qella di trattare il decadimento del protone o del neutrone in modo del tutto simile alla fisica atomica. Così come un atomo può passare da uno stato quantico a un altro emettendo o assorbendo un fotone
            \begin{equation*}
                \ce{A^* -> A + \gamma}
                \mycomma
            \end{equation*}
            il protone e il neutrone sono trattati stati quantici distinti della stessa particella e l'uno può trasformarsi nell'altro attraverso l'emissione di un \betap\ e di un neutrino. Si parla quindi di un'interazione puntiforme tra quattro fermioni regolata dalla costante di Fermi $G_\text{F} = \SI{1.166}{\giga\eV (hc)^3}$ 
            \begin{align*}
                \ce{n &-> p + e^- + \overline{\nup}_{\text{e}}}\mycomma\\
                \ce{p &-> n + e^+ + \nup_{\text{e}}}
                \myperiod
            \end{align*}
            Il modello di Fermi dà anche una stima più corretta del tasso di decadimento e dello spettro di energia trasportata dagli eiettili.

            Il modello di Fermi è stato successivamente aggiustato in seguito agli esperimenti svolti da C. Rubbia, scopritore dei bosoni mediatori: il protone e il neutrone non decadono direttamente emettendo la particella \betap\ e il neutrino, ma piuttosto uno dei quark che li costituisce decade emettendo una particella intermedia---un bosone mediatore dell'interazione debole, $\Wp^\pm$ o $\Zp^0$ a seconda che l'interazione sia a corrente carica o neutra---che a sua volta decade nella coppia \betap--$\nup_\text{e}$ \pfigref{fig:beta-decay}. Questo nuovo modello porta a una ridefinizione della costante di accoppiamento, che è quindi data da
            \begin{equation*}
                G_\text{F} = \sqrt{2} \frac{g^2}{8 M_{\Wp}^2}
                \mycomma
            \end{equation*}
            essendo $g$ la nuova \emph{costante di accoppiamento debole} ed $M_{\Wp}$ la massa del bosone \Wp. 
            \begin{figure}
                \centering
                \begin{tikzpicture} [scale = 0.5]
    \coordinate (A) at (0,0);
    \coordinate (B1) at (1.5,2.5);
    \coordinate (B2) at (2.5,1.5);
    \coordinate (B3) at (-1,2.5);

    \begin{scope} [decorate, decoration={markings, mark = at position 0.5 with {\arrow{>}}}]
        \draw[postaction = {decorate}] (-1,-2.5) node [below] {\tiny n} -- (A);
        \draw[postaction = {decorate}] (A) -- (B1) node [above] {\tiny $\overline{\neutrino}_\electron$};
        \draw[postaction = {decorate}] (A) -- (B2) node [above] {\tiny $\hphantom{^-}\electron^-$};
        \draw[postaction = {decorate}] (A) -- (B3) node [above] {\tiny p};
    \end{scope}
\end{tikzpicture}
                \hspace{1cm}
                \begin{tikzpicture} [scale = 0.5, every node/.style = {font = \tiny}]
    
    \coordinate (A1)    at (-2,-1.5);
    \coordinate (A)     at (-1,0);
    \coordinate (A2)    at (-2,1.5);
    \coordinate (B1)    at (2,1.5);
    \coordinate (B)     at (1,0);
    \coordinate (B2)    at (2,-1.5);

    \foreach \x/\t/\i/\f in {0/0.7/\quarkd/\quarku,0.25/0.4/\quarkd/\quarkd,0.5/0.4/\quarku/\quarku}{
    \begin{scope}[decorate, decoration={markings, mark = at position 0.5 with {\arrow{>}}}]
        \draw[postaction = {decorate}] ($(A1)-(\x,0)$) node [above = -10pt] {\i} -- ($(A)-(\x,0)$);
        \draw[postaction = {decorate}] ($(A)-(\x,0)$)   -- ($(A2)-(\x,0)$) node [above = -2pt] {\f};
    \end{scope}
    }
    \node at ($(A1)+(-0.25,-0.2)$) [below] {\neutron};
    \node at ($(A2)+(-0.25,0.2)$) [above] {\proton};
    \draw[decorate, decoration = snake] (A) -- (B);
    
    \begin{scope}[decorate, decoration={markings, mark = at position 0.5 with {\arrow{>}}}]
        \draw[postaction = {decorate}, line width = 0.4 pt] (B1)    -- (B);
        \draw[postaction = {decorate}, line width = 0.4 pt] (B)     -- (B2);
    \end{scope}
    
    \node at (0,0)  [below] {$\hphantom{^-}\Wm$};
    \node at (B1)   [above] {$\antineutrinoe$};
    \node at (B2)   [below] {$\hphantom{^-}\electronm$};
\end{tikzpicture}
                \caption{A sinistra la rappresentazione del decadimento \betap\ secondo il modello a contatto di Fermi, a destra la rappresentazione di Feynman.}
                \label{fig:beta-decay}
            \end{figure}

    \section{Il neutrino}
        Il neutrino è una particella estremamemnte elusiva con una sezione d'urto $\sigma \sim \SI{e-44}{\centi\meter\squared}$ e ha caratteristiche molto diverse da quelle delle altre particelle rivelate attraverso la fisica dei raggi cosmici. I neutrini interagiscono solo attraverso la gravità e l'interazione debole, motivo per cui è difficilissimo intercettarli e riescono ad attraversare quasi indisturbati la materia.

        Le fonti di neutrini sono prevalentemente di due tipi: la maggior parte di quelli ad alta energia proviene da sorgenti astrofisiche come le \textit{supernoavae}, i neutrini di media e bassa energia invece provengono da decadimenti nucleari e sono generati in maggiore quantità nei reattori nucleari.
        \subsection{I sapori del neutrino}
            Uno dei più importanti esperimenti di rivelazione del neutrino, svolto presso il telescopio giapponese \emph{Super-Kamiokande} costruito specificamente per i neutrini, fu effettivamente in grado di rivelare il passaggio di neutrini provenienti dal Sole, solo che i neutrini rivelati erano un terzo di quelli che erano previsti dal modello teorico.
            Il modello standard prevede l'esistenza di tre \emph{sapori} di neutrino: \emph{elettronico}, \emph{muonico} e \emph{tauonico}.

            Il neutrino sembra trasformarsi dal punto di partenza al punto di arrivo, anche in base alla lunghezza del tragitto (1--100 metri = short baseline, 100--1000 metri = long baseline). Viene studiato sia in ambito particellare attraverso la fisica delle astroparticelle che attraverso lo studio dei decadimenti nucleare. Si fanno molto esperimenti anche di grandi dimensioni a cui prendono parte migliaia di scienzati.

        Rispetto al 1900 sappiamo molto di più quindi sappiamo in che direzione andare con la ricerca, ad esempio nei termini(?) di mescolamento. Ci interessano molto i neutrini di origini cosmiche che hanno energia molto elevata e possono essere rivelati solo con grandi apparati sotterranei, sottoghiaccio o sott'acqua come il km3net che ha rivelato di recente il neutrino astroparticellare più energetico mai osservato con energie del PeV. Adesso studiamo anche i neutrini che vengono dai reattori nucleari.
        
        % Gli sperimentali, i nucleari e altri fisici studiano queste cose e fanno i conti alla magistrale e al dottorato (deduco che noi non li facciamo). Ci sono scuole di alta formazione che hanno come unico obiettivo quello di approfondire la fisica del neutrino.

    \subsection{Il neutrino e i fisici}
        All'epoca di Fermi il neutrino non veniva studiato a ``compartimenti stagni''. I fisici dell'epoca erano a tutto tondo, sviluppavano la teoria e immaginavano, costruivano ed eseguivano esperimenti. Ad esempio Fermi con i raggi cosmici ad alta energia, che immaginò potessero esse generati da esplosioni di supernove.

        Dal 2010 invece lo studio della fisica del neutrino ha iniziato a differenziarsi, le sottoaree sono state assorbite dai gruppi di scienziati dell'ambito in cui viene osservato: a Catania sono divisi in nucleari, astroparticelle e acceleratori.

        % Fermi immaginò che l'interazione del neutrino fosse un'interazione a 4 fermioni. Introdusse quella che oggi chiameremmo una costante di accoppiamento $G_\text{F}$. Naturalmente si tratta di una possibile interpretazione che seppur non sia del tutto corretta comunque ha costituito un primo modello. Non potevano fare simulazioni e non avevano strumenti per conservare i dati come oggi.

        % L'interpretazione moderna di questa interazione d icontatto dei 4 fermioni è un po' meglio. 

        % Il neutrino è presente laddove c'è interazione debole, attraverso interazioni di corrente carica e corrente neutra. Il neutrino interagisce solo attraverso interazione debole. Vedremo l'interazione debole a parte. Essendo neutro il neutrino non può interagire col campo elettromagnetico, Pauli prevede massa nulla ma Fermi---che aveva ragione---pensava avesse una massa molto piccola, neutro come un neutrone, per questo essendo un piccolo neutrone decide di battezzarlo neutrino.

        % Gli esperimenti di Carlo Rubbia (Nobel) al CERN confermò l'esistenza dei prpagatori dell'interazione debole: $W^+$, $W^-$ e $Z^0$.

        Albergo cita sempre questi esperimenti e li chiede agli esami quindi li chiede anche lei per continuità!
    \subsection{Il neutrino nel modello standard}
        Il neutrino, come già detto, è parte delle particelle elementari che non sono da confondere con le particelle subatomice. Le particelle elementari sono i quark e i leptoni e per quanto ne sappiamo oggi son indivisibili. I protoni sono particelle subatomiche ma non sono elementari, essendo costituite da agglomerati di tre quark.

        Le interazioni vengono rappresentate attraverso altre particelle mediatrici come i fotoni per l'interazione elettromagnetica, i bosoni $W^\pm$, $Z^0$ per l'interazione debole, i gluoni per l'interazione forte e il bosone di Higgs che giustifica l'esistenza della massa.

        Ai neutrini viene attribuita massa nulla in maniera artificiosa, per far ``tornare i conti'' del modello standard, ma attraverso il metodo scientifico l'ipotesi è stata smentita attraverso la scoperta e lo studio delle oscillazioni. Lo studio delle oscillazioni è qualcosa che non viene previsto dal modello standarad quindi è \emph{beyond the standard model}. Matematicamente parlando, le oscillazioni sono interpretate come sovrapposizioni di autostati di massa distinti, e l'esistenza di massa prevede che il neutrino abbia massa, in contraddizione col modello standard. Al momento non abbiamo un numero per la massa ma solo dei limiti superiori, sappiamo che la massa del neutrino è non nulla e minore di una certa quantità. La massa del neutrino cambia come quella dei quark e degli altri leptoni cambiando famiglia, in ordine elettronico < muonico < tauonico.

        La sua sezione d'urto è di \SI{10e-44}{\centi\meter\squared} e di conseguenza servono target estremamente massivi per poter avere un'interazione. Al gran sasso si usano rivelatori grandi come palazzi per poter osservare l'oscillazione del neutrino. km3net usa masse d'acqua enormi. L'esperimento Auger, di cui la prof Caruso fa parte, usa rivelatori vicino alle Ande e fa rivelazioni di fotoni e neutrini astroparticellari. Non essendoci grandi masse da sfruttare hanno rivelatori che tappezzano il terreno. In realtà misurano i neutrini che vengono dal lato opposto della terra che trapassano tutto il pianeta e interagiscono col terreno. Quindi anche se ancora non capiamo il neutrino, sappiamo che ci sono molti modi di rivelarli grazie agli avanzamenti della fisica delle particelle.
    \subsection{Decadimento beta}
        Nei fenomeni nucleari, il decadimento beta produce neutrini:
        \begin{equation*}
            n \rightarrow p + e^- + \overline{\nu}_e
        \end{equation*}
        Il neutrone è formato da tre quark u d d che si trasformano in un u u d attraverso l'emissione di un bosone W-. Il bosone, appena lasciato il nucleone, decade in un elettrone e in un antineutrino. Avendo generato un elettrone, l'antineutrino deve essere di tipo elettronico. Se venisse emesso un muone allora il neutrino si presenterebbe anch'esso di tipo muonico.
\section{Storia del neutrino}
        Carrellata di storia:
        \begin{enumerate}
            \item[1956:] Reines \& Cowan rivelazione della prima interazione diretta di (anti)-neutrino elettronico;
            \item[1962:] Danby et al prima rivelazione del neutrino muonico;
            \item[1975:] Perl scopre il leptone $\tau$;
            \item[2000:] La collaborazione DONUT del Fermilab registra il primo segnale proveniente da un neutrino tauonico.
        \end{enumerate}
        Ora torniamo indietro.
        \subsection{Il pomo della discordia}
            Pauli suppone che il neutrino abbia massa nulla, Fermi suppone che abbia una massa dell'ordine di grandezza del centesimo della massa dell'elettrone. Il modello standard attribuisce massa nulla. Dagli anni '40 agli anni '80 sono stati cercati diversi metodi per misurare la massa, che nel complesso si possono dividere in due grandi classi: le misure \emph{dirette} e le misure \emph{indirette}. A loro volta le misure dirette si dividono in filoni: studi cinematici in laboratorio---decadimenti, pochi ricercatori, faccio decadere cose e uso la meccanica classica e relativistica per studiare il moto del neutrino---, studio del doppio decadimento beta---senza neutrini, indicato dalla sigla $\betap\betap_0$---e infine neutrini da esplosione da supernovae.

            Le misure indirette invece si basano prevalentemente sullo studio delle oscillazioni del neutrino nella materia e sono metodologie più moderne. Quello che si va a misurare però non è la massa, ma piuttosto la differenza al quadrato $\Delta m^2$ degli autostati di massa dell'Hamiltoniana del neutrino secondo la teoria delle oscillazioni.

        \subsection{Misure dirette}
            \begin{enumerate}
                \item Fermi--Kurie plot: Fermi ipotizza che l'attribuzione di massa al neutrino influenzerebbe la parte finale---\emph{end-point}---dei grafici del decadimento beta. In questi grafici è rappresentato lo spettro di emissione in funzione dell'energia e con esperimenti molto precisi si osserva una curvatura nell'estremo destro dello spettro. Vai a vedere i grafici! Il succo è che l'andamento della curva sarebbe molto diverso se il neutrino non avesse massa, e basta una massa molto piccola a cambiare la forma dell'\emph{end-point}. Un problema consistente è che nell'\emph{end-point} del grafico cadono circa \num{10e-10} decadimenti del totale, che sono veramente pochi. Il limite superiore della massa ottenuto da questa misura è di \SI{10}{\eV}.
                
                Si usano decadimenti con un \Qv\ piccolo per evitare che il neutrino, che ha massa molto bassa, venga nascosto da un'energia di decadimento molto alta. Più è piccolo il \Qv meglio viene la misura. Serve inoltre un tempo di dimezzamento breve per avere molti decadimenti e poter fare più misure. Può infine essere utile avere una struttura nucleare semplice, dal momento che il decadimento è influenzato dall'interazione coulombiana e avere un nucleo piccolo e semplice introduce meno errori. Un ottimo nucleo che si presta a questo tipo di eperimento è il trizio che decade in elio 3 + elettrone + antinuetrino [mettere nella formula chimica]. Il trizio ha un tempo di dimezzamento di 12.3 anni.

                \item Esperimenti da decadimenti deboli: si sfruttano i decadimenti del pione in muone + neutrino muonico. oppure dal decadimento del tauone in tre pioni e un neutrino tauonico. Si sparano fasci di protoni su un target spaccando i nuclei del materiale e formando uno sciame di adroni e altre particelle secondarie, in modo simile agli sciami originati da raggi cosmici. Di conseguenza non li chiamo \emph{extensive air shower} ma solo \emph{shower}. La valanga di particelle infine attraversa dei rivelatori che sono in grado di risalire alla traiettoria del proiettile---rivelatori \emph{tracker}---e all'energia delle particelle---calorimetri formati da un mezzo denso e un mezzo scintillante, circondati da fotomoltiplicatori che convertono l'energia immessa nello scintillatore in segnale elettrico.
                
                Affinché questi esperimenti siano utili è necessario riconoscere tutte le interazioni che possono avvenire a valle. Comunque nello sciame secondario possono esserci pioni neutri o pioni carici, quelli neutri decadono in due gamma mentre quelli carichi decadono in coppie di muoni e neutrini. Quello che si fa è quindi generare sciami che contengano pioni e osservare i neutrini emessi dal suo decadimento. da qui massa del neutrino munoico minore di \SI{120}{\kilo\eV} %controllare!!

                \item Lo stesso può essere fatto osservando il decadimento dei $\tau$ che emettono tre pioni e un neutrino tauonico. [Vedi slide per dettagli sugli sperimentatori e l'anno]. I tauoni vengono generati nei collider presso il DORIS (1988) e il CERN (1995) trovndo una massa minore di \SI{24}{\mega\eV}
            \end{enumerate}
            Scala di massa? I limiti superiori ci fanno capire che le masse dei diversi sapori di neutrino sono di ordini di grandezza diversi.
            \begin{enumerate}
                \item Il doppio decadimento $\betap\betap_0$ può essere diverso in base al tipo di neutrino che può essere distinto in neutrino di Dirac o di Majorana. Goldhaber, Grozins e Sunyar osservano una reazione strana [vedi slide] dove il neutrino ha la stessa elicità di qualcosa boh?? di che sta parlando.? Ci sono neutrini destrorsi e sinistrorsi se immaginiamo l'elicità come una rotazione intorno alla direzione di propagazione. Ricordi l'esperimento di madame Wo dove c'era l'elicità dell'elettrone? Ecco, una cosa del genere.
                
                Pauli introduce uno spin di $1/2$ per bilanciare il momento angolare nel decadimento $\betap$. Possiamo dare l'elicità del neutrino come:
                \begin{equation*}
                    \eta = \frac{\vb{S}\cdot\vb{p}}{\norm{\vb{S}}\norm{\vb{p}}}
                \end{equation*}
                Se per un attimo immaginiamo lo spin come una rotazione della particella su sé stessa. Dall'eseprimento si osserva che l'elicità del neutrino è sempre \num{-1}, quindi viene detto particella \emph{sinistrorsa} o \emph{left-handed}. Se applichiamo il teorema di conservazione di carica, parità e tempo allora prevediamo che l'antineutrino possa essere \emph{destrorso} o \emph{right-handed}.

                Qualcosa sulla conservazione della coniugazione di carica? [rec a 38 minuti]

                
            \end{enumerate}bbbb
            Distinguiamo il neutrino di Dirac da quello di Majorana in base alle idee proposte. Dirac propone che il neutrino e l'antineutrino sono distinti, il neutrino è sempre \emph{sinistrorso} e l'antineutrino è sempre \emph{destrors}: non possono esistere le configurazioni invertite, che vengono dette \emph{sterili}. Dirac afferma inoltre che si deve conservare il numero leptonico, quindi:
            \begin{align*}
                \pi^+ &\rightarrow \mu^+ + \nu_\mu\\
                \pi^- &\rightarrow \mu^- + \overline{\nu}_\mu\\
            \end{align*}
            e inoltre
            \begin{align*}
                \nu_\mu + N &\rightarrow \mu^- + \text{had.}\\
                \overline{\nu}_\mu + N &\rightarrow \mu^+ + \text{had.}
            \end{align*}
            [sentire rec]
            Il neutrino di Majorana invece è molto diverso dalla teoria di Dirac. Per lui il neutrino e l'antineutrino sono la stessa particella---segue banalmente che sono possibilie sia neutrini sinistrorsi che destroris---e sostiene che il numero leptonico non debba conservarsi:
            \begin{align*}
                \pi^+ &\rightarrow \mu^+ + \nu_\mu^-\\
                \pi^- &\rightarrow \mu^- + \nu_\mu^+
            \end{align*}
            essendo $\nu_\mu^+$ il destrorso e il $\nu_\mu^-$ il sinistrorso.

            Con i nostri esperimenti cerchiamo di capire se i neutrini sono come previsto da Dirac o come da Majorana. I fattori necessari per poter distinguere i due tipi di neutrini sono:
            \begin{enumerate}
                \item Una sorgente di neutrini come quelli che abbiamo visto fin'ora, insieme a rivelatori come calorimetri e rivelatori Cherenkov.
                \item La possibilità di invertire artificialmente l'elicità del neutrino nel corso dell'esperimento
                \item Studiare l'interazione del neutrino con elicità invertita: se il neutrino invertito non interagisce vuol dire che non esiste o è sterile come previsto da Dirac e viceversa se riesco ad osservarlo, secondo Majorana dovrebbe avvenire la
                \begin{equation}
                    \nu_\mu^+ + N \rightarrow \mu^+ + \text{had.}
                \end{equation}
            \end{enumerate}
            Il limite di questo programma è che fin'ora si è dimostrato impossibile invertire l'elicità del neutrino, per questo si ricercano i neutrini sterili nell'ambito della fisica astroparticellare o nei doppi decadimenti $\betap\betap_0$.


%====================%
%                    %
%   29 MAGGIO 2025   %
%                    %
%====================%

\section{29 maggio 2025}
Torniamo alla questione neutrino di Dirac e neutrino di Majorana.
\subsection{esperimento concettuale e doppio decadimento \betap}.
            Il decadimento \betap\ è un decadimento con emissione di un neutrino, SE ESISTONO, i doppi decadimenti \betap\ dovrebbero emettere due neutrini(?) si scrive come $\betap\betap_0$ perché nel complesso non viene emesso il neutrino
            \begin{equation}
                \pqty{A,Z} -> \pqty{A,Z+2} + e^- + e^-
            \end{equation}
            Se un nucleo decade \betap\ emette un elettrone e un neutrino. Un doppio \betap\ permesso dovrebbe essere un decadimento con 2 elettroni e 2 neutrini. Ma può esistere un doppio \betap\ dove non viene emesso nessun neutrino? o meglio dove un neutrino viene emesso e riassorbito e restano solo i due elettroni. Questo tipo di decadimento sarebbe possibile se l'antineutrino coincidesse con il neutrino stesso e avesse massa non nulla.

            In che modo permette di distinguere tra Dirac e Majorana? Il doppio decaimento senza neutrino non è possibile per i neutrini di Dirac ma solo per quelli di Majorana. In ogni caso è un processo del secondo ordine, ovvero sfavorito probabilisticamente e che avviene solo quando un decadimento del primo ordine come il classico \betap\ è proibito energeticamente.

            Nel caso di Majorana, siccome non si osserva il neutrino, parliamo di neutrino virtuale che viene scambiato (tipo i fotoni virtuali dell'interazione EM?). Ora noi abbiamo appurato che il neutrino ha massa non nulla.

            Kamiokande e osservatorio del gran sasso (MACRO, esperimento completamente diverso! quindi conferma molto bene) avevano scoperto le oscillazioni del neutrino che sono possibili solo se la massa è non nulla.
            
            La reazione divisa in due step sarebbe:
            \begin{align}
                \pqty{A,Z} -> \pqty{A,Z+1} + e^- + \nup \\
                \nup\pqty{h = -1} + \pqty{A,Z+1} -> e^- + \pqty{A,Z+2}
            \end{align}
            questo processo richiede uno spin fotomoltiplicatori
            Sono importanti dei punti fondamentali: l'ampiezza di probabilità per produrre un neutrino denna lprima transizione è pari a $m_\nup / E_\nup$, quindi quella per il doppio decadimento è proporzionale al quadrato di $m_\nup$. Chiaramente se la massa del neutrino è nulla la probabilità di osservarlo
            
            che esperimenti sono stati fatti su questa cosa?
            \begin{equation*}
                \ce{_{32} Ge ^{76} -> _{34}Se^{76} + e^- + e^-}
            \end{equation*}
            c'è un esperimento che si chiama CUORE fatto da Fiorini e co. (in corso, è l'esperimento di riferimento sul $\betap\betap_0$) sottoterra underground in silenzio cosmico (quindi bloccando tutta la radiazione cosmica), le condizioni devono essere super controllate e studia dei decadimenti rarissimi. L'apparato è grande quanto un palazzo di 4 piani e prima è stata testata larealizzabilità dell'esperimento con un prototipo chiamato CUORICINO. Servono fisici delle particelle e delle astroparticelle e i fenomeni vengono studiati a temparture prossime a \SI{0}{\kelvin}. Ma a temperature basse succedono cose strane! già i sipm a -40 gradi celsius e se magari il fotosensore resiste grazie a delle celle climatiche ma la colla e tutta la parte strutturale e l'elettronica e cavi cedono a basse temperature! Se vado sotto figurati!

            Comunque Fiorini nel primo esperimento (non cuore) faceva decadere germanio in selenio e non leggeva nessun segnale. L'esperimento MACRO era per trovare monopoli magnetici e il fatto di non averli trovati permette di dare un limite superiore a quanti se ne possono vedere (sicuramente meno di quelli che lo strumento può rivelare, anche se dovessero essere 0). Quindi sostanzialmente diedero un limite superiore alla vita media del germanio e alla massa del neutrino. Maier (1994) e Heidelberg--Mosca (1998) danno una nuova stima migliorata.

            In seguito l'esperimento NEMO-3 nei laboratori sotterranei del Frejus (2003) dà risultati molto migliori (vedi slide per i numeri).

            Anche il NEMO è estremamemtne complicato (ci sono delle belle foto!). Da questi esperimenti (che sono misure dirette) possiamo iniziare a tirare giù delle tabelle su caratteristiche e limiti superiori.

        \subsection{Aspe}
            gli esperimenti da misure indirette danno stime migliori sulla massa del neutrino, poi le facciamo!
        \subsection{Sorgenti}
            Sorgenti! Naturali:
            \begin{enumerate}
                \item neutrini terrestri da radioattività ambientali
                \item neutrini atmosferici da raggi cosmici
                \item neutrini solari (catene pp, CNO)
                \item neutrini da esplosione di supernovae
                \item neutrini dal Big Bang 
                \item neutrini da interazioni dei raggi cosmici primari tra loro nello spazio intergalattico
            \end{enumerate}
            Di quelle artificiali abbiamo:
            \begin{enumerate}
                \item neutrini ad reattore
                \item neutrini da acceleratore
            \end{enumerate}
            Non si possono ovviamente fare esperimenti in grado di rivelare tutti i tipi di neutrini a tutte le possibili energie, per questo ogni ``tipologia'' di neutrino deve essere studiata con strumenti specifici. Quello che si può fare in seguito è combinare i dati nei grafici di esclusione e estrapolare più informazioni possibili. Vediamo ora le sorgenti di origine naturale:
            \begin{enumerate}
                \item[\textbf{Terrestri.}] Al gran sasso rivelatore borexino bellissimo! foto belle! fa neutrini solari, fa la stessa cosa del kamiokande ma consapevolmente (sa che ne deve beccare un terzo). Questo rivelatore becca anche neutrini terrestri dal decadimento di elementi come Uranio e Torio. Sono rivelati dal superkamiokande e dal borexino. I modelli attuali sulla struttura interna della terra sono solo teorici. Juno forse aiuterà: Un altro grande osservatorio che fa questa cosa à JUNO, stessa cosa, una palla di acqua con fotomoltiplicatori. è in una caverna in cina circondato da una decina di reattori nucleari e cerca di rivelare proprio quelli, oltre che quelli atmosferici, da supernove e dal core terrestre. Ci sono stati 10 anni per pensarlo e metterlo a punto, qualche mese fa è stata riempita la sfera di scintillatore liquido e ora inizierà la presa dei dati. Perché il neutrino è un \emph{messaggero del cosmo}? è una sonda potentissima ma che significa????? beh!!!! ci porta informazioni dal core della terra e ai geofisici torna molto utile averlo come strumento. è meglio di fare carotaggi, non possiamo mica andare sottoterra.  
                
                \item[\textbf{Atmosferici.}] Per quel che riguarda i neutrini atmosferici, sono quelli prodotti durante le interazioni che avvengono nelle \textit{Extensive Air Shower} (\SI{e+21}{\eV}). Sono generate da raggi cosmici e altre particelle cariche ad alte energie sparate dal Sole. Se ne formamo molti di tipo muonico e tauonico e contninuano a formarsene finché lo sciame riesce a produrre nuove particelle. Alcune particelle arrivano a terra e vengono dette penetranti. Alcuni muoni viaggiano per 300 metri sottoterra ma i neutrini molto di più! questi esperimenti permettono di misurare sezioni d'urto, energiem oscillazioni etc. sta ripetendo le stesse cose sul fatto che oscillano e ne vediamo di meno. quelli atmosferici sono utili a studiare le proprietà dei neutrini per poi rivelare meglio quelli che vengono da altre sorgenti.
                
                \item[\textbf{Solari.}] I neutrini solari sono proposti da Bethe nel 1938 con un modello di reazioni nucleari nel sole. Il ciclo pp unisce due protonie e con un deacdimento \betap\ si produce un deuterio etc... Ci hanno permesso di studiare le reazioni nucleari che avvengon oall'interno del sole. Anche qui da catene pp e CNO (molti meno) ci aspettiamo tot neutrini dal modello stadard del sole di (bacoll??) e se ne trovavano sempre di meno (vedi kamiokande). ovviamente i fisici solari credevano al modello standard del sole mentre i particellari credevano alle loro misure. Mancavano perché i neutrini elettronici attraversando il sole estremamente denso si trasformano e diventano muonici e tauonici. Quindi giusto che ne vediamo di meno! tra l'altro nella materia si innesca un fenomeno di risonanza e i neutrini oscillano maggiormente nella materia.
            
                Vedere gli schemini, il Borexino ha dato misure precisissime sulla catena pp ma anche per i decadimenti di Berillio e altre reazioni e decadimenti.
                
                Che problemi abbiamo coi neutrini solari? Il Sole produce un flusso di \SI{6e+10}{\per\second\per\centi\meter\squared} neutrini, veramente tanti! Questo flusso è stato misurato in diversi contesti e con diversi energie, e tutti gli esperimenti misuravano più o meno la metà dei neutrini attesi. Stiamo parlando degli anni 60, ci sono voluti 40 anni per scoprire che erano le oscillazioni. Il modello solare di bacoll è confermato!
                
                \item[\textbf{Da Supernovae.}] Le stelle \textbf{SCHIATTANO!!! PAMMMMM} producono un sacco di neutrini con delle caratteristiche specifiche. Sono VERAMENTE VERAMENTE energetici. CI permettono di studiare le stesse supernove e ci sono molti astrofisici che studiano le supernove quindi è una sorgente più o meno conosciuti. C'è l'esperimento LWD che è stato costruito dopo la famosa supernova nel 1987-A. Purtroppo da noi ce n'è una ogni 30 anni, magari a preve ne scoppia una! Speriamo in JUNO e in LWD.
                
                In seguito all'esplosione del 1987 nella nube di Magellano sono stati  rivelati 11 neutrini in 10 secondi da due diversi rivelatori che cercavano neutrini solari. Questi neutrini venivano dalla povera stella scoppiata (R.I.P. sellina) ma questo confermava il modello teorico delle Supernovae.

                A noi interessa in particolare capire in quale momento del collasso stellare avviene l'emissione dei neutrini da Supernovae, si nota che arriva un burst di neutrini e posso misurare con quanto ritardo arrivano rispetto alla luce. Da questo ritardo posso ricavare direttamente la massa, si è ottenuto un limite superiore di \SI{12}{\eV}. Gli altri metodi non riescono a ottenere risultati più precisi, ma non possiamo aspettare 30 anni ogni volta!
                % +
                % m

                \item[\textbf{Cosmogenici.}] Sono i neutrini originati dall'interazione tra raggi cosmici e altra materia intergalattica---non l'interazione con l'atmosfera---e si tratta dei neutrini a più elevate energie conosciute. È possibile che provengano dai nuclei galattici attivi (AGN). Sono stati rivelati da Ice Cube e km3net, famosi!
                
                \item[\textbf{Cosmologici.}] Si ipotizza l'esistenza di un fondo di neutrini primordiali dal big bang, sarebbero come il fondo cosmico a microonde, che dovrebbero permeare l'universo. Li cerchiamo ma non abbiamo conferma, servono tecniche molto raffinate ed esperimenti molto diversi da quelli fatti fin'ora. è un campo aperto ma molto recente.
                 
                Secondo le ipotesi sarebbero stati prodotti moltissimi neutrini durante il Big Bang e dovrebbe essere rimasto la \emph{radiazione fossile di neutrini} di neutrini e antineutrini. Questi neutrini dovrebbero aver perso moltissima della loro energia nell'espansione dell'universo, giungendo a \SI{e-9}{\eV}, e sono per questo difficilissimi da notare nonostante siano distribuiti in modo isotropo con una densità $n \sim \SI{300}{\per\centi\meter\squared}$.
            \end{enumerate}
            È di grande utilità, soprattutto per la comprensione delle oscillazioni, lo studio dei neutrini da sorgenti artificiali:
            \begin{enumerate}
                \item[\textbf{Dal nucleare.}] I reattori nucleari vengono usati come sorgenti. Ci si mette a 10--20 metri fino a max 1km, per questo sono di tipo \textit{short baseline}. Serve conoscere bene cosa il reattore produce per poter prevedere il comportamento del neutrino lungo il tragitto. Si va a vedere il flusso a valle e lo si confronta col flusso a monte per avere informazioni sulle oscillazioni e sulla massa.
                
                Nei reattori nucleari l'uranio 235 decade alla grande e si autoalimenta producendo energia e causando altri decadimenti. I nuclei figli dell'uranio, Kripton e Bario, hanno troppi neutroni per sopravvieve e decadono \betap. Il risultato sono neutrini. Un reattore che produce 3 GW di energia produce 6 per 10 alla 20 neutrini al secondo su tutto l'angolo solido.

                L'osservatorio JUNO ha anche delel appendici fatte apposta per queste misure. Ad esempio TAU che è una palla di SiPM di qualche decimetro di diametro e misura i neutrini dai reattori nucleari.
                
                \item[\textbf{Da acceleratori.}] Più o meno la stessa cosa si fa con gli acceleratori. Qui mettiamo rivelatori da 100km a 1000km di distanza e ricerchiamo allo stesso modo i fenomeni di oscillazioni.
                
                Funziona accelerando protoni---puri e guidati come necessario---che poi producono altre particelle come kaoni e pioni. Metto un materiale assorbitore denso come il piombo che arresti i muoni e restano i neutrini che attraversano distanze lunghe. Ad esempio posso produrre nuetrini al CERN e misurarli al Gran Sasso con l'esperimento Opera che ha verificato l'esistenza del neutrino tau e delle oscillazioni.

                Questi esperimenti sono chiaramente long-baseline
            \end{enumerate}
                
    \section{Come si misurano i neutrini}