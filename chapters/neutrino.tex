\chapter{La Fisica del Neutrino}
    \section{Il problema del decadimento \texorpdfstring{\betap}{beta}}
        Il decadimento \betap\ è un fenomeno nucleare che avviene quando un nucleo instabile si trasforma in uno più stabile trasformando un protone in neutrone o viceversa; questa trasformazione viene tipicamente osservata attraverso l'emissione di una particella $\betap^\pm$, ovvero un elettrone ($\betap^-$) o un positrone ($\betap^+$).

        Il problema così impostato tuttavia dovrebbe portare all'osservazione di particelle \betap\ monoenergetiche, in particolare con un'energia cinetica pari al difetto di massa del nucleo. Nel caso del decadimento $\betap^-$ il comportamento previsto applicando il modello a due corpi è
        \begin{equation*}
            \ce{$\pqty{A, Z}$ -> $\pqty{A, Z+1}$ + e^-}
            \mycomma
        \end{equation*}
        dove $A$ è il numero di massa e $Z$ il numero atomico del nucleo. Il difetto di massa è dato dalla differenza tra la massa del nucleo iniziale e quella del nucleo finale più quella dell'elettrone: 
        \begin{equation*}
            Q_\betap = m\pqty{A, Z} - m\pqty{A, Z+1} - m\pqty*{\txt{e}^-}
            \mycomma
        \end{equation*}
        essendo $Q_\betap$ è il \Qv\ del decadimento. 
        
        Il problema è che lo spettro energetico dei \betap---osservato in un importante esperimento da Ellis e Wooster---risulta continuo, con un massimo pari al \Qv\ del decadimento, e non piccato come previsto da un decadimento monoenergetico. Per spiegare questa discrepanza, W. Pauli nel 1903 propose con una lettera l'introduzione di una nuova particella da chiamare \emph{neutrone}\footnote{Il neutrone vero e proprio sarebbe stato scoperto due anni dopo.} con massa e proprietà simili all'elettrone ma di carica neutra, che venisse emesso insieme all'elettrone per bilanciare il decadimento e preservare le leggi di conservazione.

        Dopo la scoperta del neutrone in quanto nucleone, molto più massiccio di un elettrone e che non veniva osservato come emissione nei decadimenti \betap, E. Fermi propose nel 1932--1933 di cambiare il nome della particella ipotetica in \emph{neutrino}, essendo questo molto più piccolo e sfuggente di un neutrone.

        \subsection{La proposta di Fermi per il decadimento \texorpdfstring{\betap}{beta}}
            Una delle grandi intuizioni di Fermi fu qella di trattare il decadimento del protone o del neutrone in modo del tutto simile alla fisica atomica. Così come un atomo può passare da uno stato quantico a un altro emettendo o assorbendo un fotone
            \begin{equation*}
                \ce{A^* -> A + \photon}
                \mycomma
            \end{equation*}
            il protone e il neutrone sono trattati stati quantici distinti della stessa particella e l'uno può trasformarsi nell'altro attraverso l'emissione di un \betap\ e di un neutrino. Si parla quindi di un'interazione puntiforme tra quattro fermioni regolata dalla costante di Fermi $G_\text{F} = \SI{1.166}{\giga\eV (hc)^3}$ 
            \begin{align*}
                \ce{\neutron &-> {\proton} + {\electronm} + {\antineutrinoe}}\mycomma\\
                \ce{\proton &-> {\neutron} + {\positron} + {\neutrinoe}}
                \myperiod
            \end{align*}
            Il modello di Fermi dà anche una stima più corretta del tasso di decadimento e dello spettro di energia trasportata dagli eiettili.

            Il modello di Fermi è stato successivamente aggiustato in seguito agli esperimenti svolti da C. Rubbia, scopritore dei bosoni mediatori: il protone e il neutrone non decadono direttamente emettendo la particella \betap\ e il neutrino, ma piuttosto uno dei quark che li costituisce decade emettendo una particella intermedia---un bosone mediatore dell'interazione debole, \Wpm o \Z a seconda che l'interazione sia a corrente carica o neutra---che a sua volta decade nella coppia \betap--\neutrinoe \pfigref{fig:beta-decay}. Questo nuovo modello porta a una ridefinizione della costante di accoppiamento, che è quindi data da
            \begin{equation*}
                G_\text{F} = \sqrt{2} \frac{g^2}{8 M_{\W}^2}
                \mycomma
            \end{equation*}
            essendo $g$ la nuova \emph{costante di accoppiamento debole} ed $M_{\W}$ la massa del bosone \W. 
            \begin{figure}
                \centering
                \begin{tikzpicture} [scale = 0.5]
    \coordinate (A) at (0,0);
    \coordinate (B1) at (1.5,2.5);
    \coordinate (B2) at (2.5,1.5);
    \coordinate (B3) at (-1,2.5);

    \begin{scope} [decorate, decoration={markings, mark = at position 0.5 with {\arrow{>}}}]
        \draw[postaction = {decorate}] (-1,-2.5) node [below] {\tiny n} -- (A);
        \draw[postaction = {decorate}] (A) -- (B1) node [above] {\tiny $\overline{\neutrino}_\electron$};
        \draw[postaction = {decorate}] (A) -- (B2) node [above] {\tiny $\hphantom{^-}\electron^-$};
        \draw[postaction = {decorate}] (A) -- (B3) node [above] {\tiny p};
    \end{scope}
\end{tikzpicture}
                \hspace{1cm}
                \begin{tikzpicture} [scale = 0.5, every node/.style = {font = \tiny}]
    
    \coordinate (A1)    at (-2,-1.5);
    \coordinate (A)     at (-1,0);
    \coordinate (A2)    at (-2,1.5);
    \coordinate (B1)    at (2,1.5);
    \coordinate (B)     at (1,0);
    \coordinate (B2)    at (2,-1.5);

    \foreach \x/\t/\i/\f in {0/0.7/\quarkd/\quarku,0.25/0.4/\quarkd/\quarkd,0.5/0.4/\quarku/\quarku}{
    \begin{scope}[decorate, decoration={markings, mark = at position 0.5 with {\arrow{>}}}]
        \draw[postaction = {decorate}] ($(A1)-(\x,0)$) node [above = -10pt] {\i} -- ($(A)-(\x,0)$);
        \draw[postaction = {decorate}] ($(A)-(\x,0)$)   -- ($(A2)-(\x,0)$) node [above = -2pt] {\f};
    \end{scope}
    }
    \node at ($(A1)+(-0.25,-0.2)$) [below] {\neutron};
    \node at ($(A2)+(-0.25,0.2)$) [above] {\proton};
    \draw[decorate, decoration = snake] (A) -- (B);
    
    \begin{scope}[decorate, decoration={markings, mark = at position 0.5 with {\arrow{>}}}]
        \draw[postaction = {decorate}, line width = 0.4 pt] (B1)    -- (B);
        \draw[postaction = {decorate}, line width = 0.4 pt] (B)     -- (B2);
    \end{scope}
    
    \node at (0,0)  [below] {$\hphantom{^-}\Wm$};
    \node at (B1)   [above] {$\antineutrinoe$};
    \node at (B2)   [below] {$\hphantom{^-}\electronm$};
\end{tikzpicture}
                \caption{A sinistra la rappresentazione del decadimento \betap\ secondo il modello a contatto di Fermi, a destra la rappresentazione di Feynman.}
                \label{fig:beta-decay}
            \end{figure}
    
    \section{Sorgenti di neutrini}
        Vediamo adesso quali sono le sorgenti principali di neutrini, distinguendole tra \emph{naturali} e \emph{artificiali}. Di solito si usa distinguere gli stessi neutrini dando loro un nome che dia informazioni sulla loro origine. In questo modo, i neutrini di origine naturale possono essere distinti in:
        \begin{enumerate}[label = $\star$]
            \item neutrini terrestri;
            \item neutrini atmosferici;
            \item neutrini solari;
            \item neutrini da \textit{Supernovae};
            \item neutrini cosmologici;
            \item neutrini cosmogenici.
        \end{enumerate}
        Tra quelli artificiali invece distinguiamo solo due tipi:
        \begin{enumerate}[label = $\star$]
            \item neutrini da reattore;
            \item neutrini da acceleratore.
        \end{enumerate}
        È importante ricordare che la particella in gioco è sempre la stessa, un neutrino terrestre non ha nulla di diverso da un neutrino cosmologico se non il modo in cui si è originato e il percorso che ha fatto prima di raggiungere il rivelatore.

        Scendiamo adesso un po' più nel dettaglio delle sorgenti dei neutrini appena elencati.
        \begin{enumerate}[wide = 0pt, leftmargin = 1.5\parindent]
            \item[\textbf{Neutrini terrestri.}] Si tratta di neutrini prodotti dalla Terra---per questo chiamati anche geo-neutrini. L'esistenza di neutrini prodotti dal nucleo della Terra era stata ipotizzata fin da subito ma essi sono stati rivelati solo recentemente. Il primo esperimento ad aver effettuato una misura di questi neutrini è stato \emph{Borexino} nel 2010.\footnote{Borexino nasce come esperimento per la misura di neutrini solari originati dalla fusione del boro. I dati ricavati dall'esperimento sono stati pubblicati pochi anni fa} L'apparato, che si trova nel Gran Sasso, è una sfera gigantesca piena di scintillatore liquido (trimetilborato), la cui superficie è tappezzata da fotomoltiplicatori.
            
            Il nucleo della Terra si trova a temperatura e pressione elevatissime, è ferroso e caratterizzato dalla presenza di nuclei attivi come uranio, plutonio e torio, che decadono con decadimenti \betap\ producendo neutrini. I neutrini quindi attraversano quasi indisturbati lo spessore terrestre e possono essere rivelati sulla crosta. Questi neutrini portano con sé informazioni sulla struttura interna del nostro pianeta, permettendoci quindi di comprenderne e prevederne l'evoluzione negli anni.
            
            Un altro esperimento molto importante è l'esperimento JUNO, anch'esso costituito da un'enorme sfera di scintillatore liquido e fotomoltiplicatori posizionata all'interno di una caverna in Cina nei pressi di una decina di reattori nucleari. L'obbiettivo principale di JUNO è quello di misurare neutrini prodotti dai reattori circostanti ma riesce anche a misurare neutrini atmosferici, neutrini da \textit{Supernovae} e neutrini geomagnetici.
            
            \item[\textbf{Atmosferici.}] I neutrini atmosferici sono quelli prodotti dalle interazioni in atmosfera che si sviluppano nelle \textit{extensive air shower}. Durante questi eventi viene prodotta un'enorme quantità di particelle secondarie tra cui neutrini elettronici e muonici.
            
            Mentre il resto della radiazione penetrante viene assorbito al suolo, i neutrini possono viaggiare attraverso la Terra come i geo-neutrini: quando si rivelano i neutrini atmosferici lo si fa puntando i rivelatori verso il terreno e osservando le particelle che viaggiano dal basso verso l'alto. In questo modo si esclude tutta la radiazione atmosferica proveniente dal cielo e si misurano solo i neutrini provenienti dal lato opposto del pianeta.

            A differenza di neutrini originati da altre sorgenti questi non ci danno grandi informazioni sull'universo, ma sono utili per studiare alcune proprietà dei neutrini stessi come la sezione d'urto, il libero cammino medio, la massa e le loro proprietà di oscillazione.

            \item[\textbf{Neutrini solari.}] Come dice il nome, si tratta di neutrini prodotti dal Sole, il quale si mantiene in vita attraverso le continue reazioni nucleari delle catene \proton\proton\ e del ciclo \ce{CNO}. La rivelazione di neutrini solari ha costituito il primo mezzo per studiare le caratteristiche del Sole e delle reazioni che avvengono al suo interno.
            
            I neutrini solari, proposti inizialmente da Bethe studiando le reazioni nucleari del Sole nel 1938, sono stati argomento di accesa diatriba tra fisici del neutrino e fisici del Sole, dal momento che il modello standard del Sole di Bahcall prevede un flusso di neutrini, pari a circa \SI{6e+10}{\per\second\per\centi\meter\squared}, molto più elevato di quello misurato dagli esperimenti a terra. La discrepanza era stata osservata in particolare durante l'esperimento Homestake (1984), il primo a contare neutrini solari, che misurava circa un terzo dei neutrini previsti dal modello standard.

            Ci sono quindi voluti più di quarant'anni per comprendere il problema: il modello teorico non prevedeva l'oscillazione dei neutrini---fenomeno che permette ai neutrini di cambiare sapore durante il tragitto attraverso la materia estremamente densa del Sole---e quindi l'esperimento stava effettivamente rivelando solo i neutrini del sapore previsto dalla teoria e perdendo tutti quelli che si erano trasformati negli altri sapori.

            Questo fenomeno è stato studiato con più consapevolezza, sempre per i neutrini solari, dall'esperimento Super-Kamiokande. Ad oggi, l'esperimento Borexino già nominato prima costituisce lo stato dell'arte sulla ricerca sui neutrini solari, le catene \proton\proton\ e il ciclo \ce{CNO}, avendo permesso di fare misure estremamente precise.

            \item[\textbf{Neutrini da \textit{Supernovae}.}] Dal punto di vista particellare, le \textit{Supernovae} sono sorgenti di raggi cosmici e di neutrini ad altissima energia. Attraverso lo studio dei neutrini prodotti in contesti così estremi, possiamo ottenere informazioni preziose sui processi che caratterizzano questi eventi astrofisici, oltre che sui neutrini stessi.
            
            Dai modelli astrofisici che sono ben noti e testati, sappiamo in quale momento del collasso della stella dovrebbe avvenire l'emissione di neutrini e quale dovrebbe essere la loro energia. Quello che si osserva è che i neutrini arrivano a terra a un tempo diverso rispetto a quello della radiazione elettromagnetica: dall'analisi di questo ritardo è stato possibile ricavare un limite superiore alla massa del neutrino pari a \SI{12}{\eV}.

            Purtoppo le \textit{Supernovae} sono eventi rari e non prevedibili, e ne osserviamo in media una ogni trent'anni. Nonostante ciò nel 1987 è stato possibile misurare un picco di neutrini, provenienti dalla \textit{Supernova} 1978A, caratterizzato da un tasso di undici neutrini in dieci secondi. Questo picco è stato misurato contemporaneamente da più rivelatori distanti gli uni dagli altri, come IMB e Kamiokande, e ha permesso di fare misure molto precise sulla massa del neutrino e sulle sue proprietà di oscillazione.

            Gli esperimenti che si occuperanno di rivelare questi neutrini nel futuro saranno il Large Volume Detector, costruito nel Gran Sasso dopo l'esplosione della \textit{Supernova} 1987A, il rivelatore IceCube, situato in Antartide, e il rivelatore KM3NeT, situato nel Mediterraneo. Ad essi si aggiungerà anche il rivelatore cinese JUNO una volta pronto.

            \item[\textbf{Neutrini cosmogenici.}] Sono i neutrini originati dall'interazione tra raggi cosmici e materia intergalattica---non l'atmosfera---e si tratta dei neutrini a più elevate energie conosciute. È possibile che provengano dai nuclei galattici attivi (AGN). Sono stati rivelati da IceCube e KM3NeT. 
            
            \item[\textbf{Neutrini cosmologici.}] Da non confondere con quelli \emph{cosmogenici}, i neutrini cosmologici sono puramente una speculazione teorica e non sono stati ancora osservati. Si tratterebbe di un fondo cosmico di neutrini, simile al fondo cosmico a microonde, originati proprio dal Big Bang. Questo fondo prende il nome di \emph{relic neutrino background} o \emph{radiazione fossile di neutrini}, e si tratterebbe di neutrini di energia molto bassa, dell'ordine di \SI{e-9}{\eV}. A casua della bassa energia la loro rivelazione richiede tecnologie estremamente avanzate che ancora non possediamo, nonostante la concentrazione prevista di circa \SI{340}{\per\centi\meter\squared} neutrini cosmologici.
        \end{enumerate}

        Mentre le sorgenti naturali non sono sempre prevedibili e controllabili, attraverso le sorgenti artificiali possiamo approfittare di un flusso noto e controllato di neutrini. Approfondiamo quindi le due principali sorgenti di neutrini:
        \begin{enumerate}[wide = 0pt, leftmargin = 1.5\parindent]
            \item[\textbf{Neutrini da reattore.}] I reattori nucleari producono neutrini attraverso il decadimento \betap\ di nuclei instabili. Essi producono una quantità elevatissima di neutrini: ad esempio un reattore da \SI{3}{\giga\watt} produce circa \SI{6e+20}{\per\second} neutrini elettronici su tutto l'angolo solido.
            
            Quello che si usa fare è posizionare dei rivelatori di neutrini a diverse distanze dal reattore, in via di massima nel sottosuolo, per studiare come varia il flusso di neutrini all'aumentare della distanza percorsa nella materia. Le distanze in gioco vanno da qualche decina di metri fino a circa \SI{1}{\kilo\meter}, per questo motivo vengono detti esperimenti \emph{short baseline}.

            Conoscendo bene la fenomenologia del decadimento \betap\ e le reazioni che avvengono dentro il reattore, si conosce altrettanto bene il flusso teorico di neutrini prodotti. Confrontare le misure con tale modello permette di studiare con molta precisione la massa e le oscillazioni del neutrino.

            L'osservatorio JUNO, già nominato in precedenza, avrà delle appendici dedicate alla rivelazione di neutrini da reattore come il rivelatore TAU, una sfera di qualche decimetro di diametro piena di scintillatore liquido.

            \item[\textbf{Neutrini da acceleratore.}] Come i neutrini da reattore, i neutrini da acceleratore sono prodotti da decadimenti di particelle cariche in seguito alle collisioni che vengono generate negli acceleratori di particelle. Durante queste reazioni vengono prodotti adroni e leptoni, che a loro volta decadono emettendo una grande quantità di neutrini muonici e tauonici.
            
            L'esperimento OPERA al Gran Sasso è stato il primo a misurare neutrini tauonici prodotti dall'oscillazione di neutrini muonici a loro volta prodotti al CERN. Questo tipo di esperimenti viene svolto su distanze molto lunghe, dell'ordine di \SI{e+2}{\kilo\meter}--\SI{e+3}{\kilo\meter}, e per questo vengono detti esperimenti \emph{long baseline}. 
        \end{enumerate}

    \section{Caratteristiche del neutrino}
        Il neutrino è una particella estremamemnte elusiva con una sezione d'urto $\sigma \sim \SI{e-44}{\centi\meter\squared}$ e ha caratteristiche molto diverse da quelle delle altre particelle rivelate attraverso la fisica dei raggi cosmici. I neutrini interagiscono solo attraverso la gravità e l'interazione debole, motivo per cui è difficilissimo intercettarli e riescono ad attraversare indisturbati enormi spessori di materia.

        Secondo il Modello Standard, il neutrino è una particella elementare, ovvero privo di struttura interna ed appartiene alla famiglia dei leptoni. Esso ha carica nulla e spin $1/2$, classificandosi quindi come fermione, e interagisce debolmente per mezzo dei bosoni \Wpm e \Z.
        
        Esistono tre sapori di neutrino corrispondenti alle tre generazioni della materia, ciascun sapore prende il nome dal leptone che lo accompagna: \emph{neutrino elettronico} \neutrinoe, \emph{neutrino muonico} \neutrinomu\ e \emph{neutrino tauonico} \neutrinotau. Come per tutte le altre particelle, esistono anche le antiparticelle corrispondenti, ovvero gli \emph{antineutrini} \antineutrinoe, \antineutrinomu\ e \antineutrinotau.

        Quando avviene un decadimento che rilascia un leptone, il neutrino associato viene emesso insieme al leptone stesso ed è dello stesso sapore del leptone emesso. Ad esempio nel decadimento \betap\ del neutrone, vengono emessi un elettrone e un (anti)neutrino elettronico, mentre nel decadimento del muone compare un neutrino muonico, oltre a un elettrone e un (anti)neutrino elettronico.

        Il neutrino può inoltre cambiare sapore durante il suo tragitto, fenomeno noto come \emph{oscillazione}. Questo fenomeno è dovuto alla sovrapposizione quantistica degli autostati di massa del neutrino, che in sostanza sono i suoi tre possibili sapori. La probabilità di oscillazione dipende dalla differenza di massa tra i vari autostati, dalla lunghezza del tragitto percorso e dall'energia del neutrino stesso.

        \subsection{La massa del neutrino}
            Il Modello Standard prevede che il neutrino abbia massa nulla. In effetti questa supposizione appare come una assegnazione artificiosa, fatta, in un certo senso, per ``far funzionare i conti''. Tuttavia, è stata proprio la scoperta delle oscillazioni dei neutrini a smentire questa ipotesi, dal momento che le oscillazioni sono possibili solo se i neutrini hanno massa non nulla---abbiamo appena detto che la probabilità di oscillazione dipende proprio dalla \emph{differenza di massa} tra i vari autostati di massa possibili.

            Tramite vari esperimenti, descritti più avanti, è stato possibile misurare la massa dei neutrini, che risulta essere molto piccola ma non nulla. È importante sottolineare che per i neutrini, più che la massa assoluta, disponiamo di una misura del limite superiore: sappiamo quindi che la massa di ciascun sapore di neutrino è inferiore a un certo valore ma non conosciamo il valore esatto, per quanto sia certamente diverso da zero. I valori attualmente accettati sono:
            \begin{equation*}
                m\pqty{\neutrinoe} < \SI{2.2}{\eV\per\clight\squared}
                \mycomma\quad
                m\pqty{\neutrinomu} < \SI{0.17}{\mega\eV\per\clight\squared}
                \mycomma\quad
                m\pqty{\neutrinotau} < \SI{18.2}{\mega\eV\per\clight\squared}
            \end{equation*}

    \section{Richiami sulle simmetrie}
        Prima di dare altri dettagli sulla fisica del neutrino e discutere dei modelli di Dirac e di Majorana, riprendiamo alcune nozioni sulle proprietà di simmetria della fisica delle particelle. Le tre principali simmetrie note sono:
        \begin{enumerate}[label = $\star$]
            \item[(C)] Simmetria per \emph{coniugaizone di carica};
            \item[(P)] Simmetria per \emph{inversione spaziale};
            \item[(T)] Simmetria per \emph{inversione temporale}.
        \end{enumerate}

        Facciamo un piccolo viaggio nel mondo della meccanica quantistica per capire meglio di cosa stiamo parlando. In meccanica quantistica le simmetrie sono descritte da operatori unitari che agiscono sugli stati fisici che indichiamo con $\ket{\psi}$. Questi operatori sono definiti come:
        \begin{align*}
            \hat{C} \ket{\psi} &= \ket*{\bar{\psi}}\mycomma\\
            \hat{P} \ket{\psi} &= \ket{\psi_P}\mycomma\\
            \hat{T} \ket{\psi} &= \ket{\psi_T}\myperiod
        \end{align*}
        La coniugazione di carica $\hat{C}$ trasforma una particella $\ket{\psi}$ nella sua antiparticella $\ket*{\bar{\psi}}$, l'inversione spaziale $\hat{P}$ inverte le coordinate spaziali della particella---corrisponde alla riflessione rispetto all'origine---e l'inversione temporale $\hat{T}$ inverte il verso del tempo, facendo avvenire gli eventi a ritroso.

        Si dice che un certo stato fisico è \emph{invariante} o \emph{simmetrico} rispetto a una delle operazioni di simmetria se l'applicazione dell'operatore allo stato lascia lo stato invariato, ovvero se valgono rispettivamente le seguenti:
        \begin{align*}
            \hat{C} \ket{\psi} &= \ket{\psi}\mycomma \\
            \hat{P} \ket{\psi} &= \ket{\psi}\mycomma \\
            \hat{T} \ket{\psi} &= \ket{\psi}\myperiod
        \end{align*}


        \subsection{Il neutrino di Dirac e di Majorana}
            Una questione ancora aperta nella fisica del neutrino è legata alla sua elicità. Ricordiamo che l'elicità è definita come il prodotto scalare tra lo spin $\vb{S}$ e l'impulso $\vb{p}$, normalizzato per i moduli dei due vettori:
            \begin{equation}
                \eta = \frac{\myip{\vb{S}}{\vb{p}}}{\norm{\vb{S}}\norm{\vb{p}}}
                \myperiod
            \end{equation}
            L'elicità può quindi essere positiva o negativa, a seconda che lo spin sia parallelo o antiparallelo all'impulso.

            L'anno successivo all'osservazione della rottura di simmetria di parità nell'esperinento di Madame Wu (1956), i fisici Goldhaber, Grodzins e Sunyar proposero il primo esperimento di misura dell'elicità del neutrino attraverso la misura della polarizzazione di raggi \photon\ emessi in seguito a un decadimento E.C. dell'europio:
            \begin{equation*}
                \ce{{^{152}Eu} + {\electronm} -> {^{152}Sm^*} + {\neutrinoe}}
                \myperiod
            \end{equation*}
            Il samario \ce{^{152}Sm^*} prodotto in questo decadimento è un nucleo eccitato che emette raggi \photon\ polarizzati,
            \begin{equation*}
                \ce{{^{152}Sm^*} -> {^{152}Sm} + \photon}
                \mycomma
            \end{equation*}
            la cui polarizzazione è legata all'elicità del neutrino emesso, per conservazione del momento angolare. Questo esperimento ha dato come risultato sempre un'elicità negativa.

            Dirac e Majorana propongono due modelli diversi per il neutrino. secondo Dirac, il neutrino è una particella distinta dalla propria antiparticella, ovvero il neutrino e l'antineutrino non coincidono. In particolare il neutrino ha sempre elicità $\eta = -1$, si dice quindi \emph{sinistrorso}, e l'antineutrino ha sempre elicità $\eta = +1$, ed è detto \emph{destrorso}. La teoria di Dirac prevede inoltre che il numero leptonico debba sempre conservarsi. Sono quindi possibili i decadimenti del tipo:
            \begin{align*}
                \ce{\pionp &-> {\muonp} + {\neutrinomu}}\mycomma\\
                \ce{\pionm &-> {\muonm} + {\antineutrinomu}}\myperiod
            \end{align*}
            Le configurazioni incrociate, ovvero quelle in cui il neutrino ha $\eta = +1$ e l'antineutrino ha $\eta = -1$, sono dette \emph{sterili}. In questo caso il neutrino non può interagire con la materia ed essere rivelato, quindi è come se non esistesse.\footnote{\authnote\ Sarò sincero, non ho capito niente di questa parte.}

            La proposta di Majorana afferma invece che il neutrino coincide con la propria antiparticella e possa quindi avere elicità $\eta = \pm1$. Secondo questa teoria il numero leptonico non si conserva e 
                    %==================%
                    %   DA SISTEMARE   %
                    %==================%
    
    
    \newpage
    \noindent\begin{minipage}{\textwidth}    
        \begin{center}
            \huge\bfseries{-- ANCORA DA SISTEMARE --}
        \end{center}
        \hrule
    \end{minipage}
        Rispetto al 1900 sappiamo molto di più quindi sappiamo in che direzione andare con la ricerca, ad esempio nei termini(?) di mescolamento. Ci interessano molto i neutrini di origini cosmiche che hanno energia molto elevata e possono essere rivelati solo con grandi apparati sotterranei, sottoghiaccio o sott'acqua come il km3net che ha rivelato di recente il neutrino astroparticellare più energetico mai osservato con energie del PeV. Adesso studiamo anche i neutrini che vengono dai reattori nucleari.
        
    \subsection{Il neutrino e i fisici}
        All'epoca di Fermi il neutrino non veniva studiato a ``compartimenti stagni''. I fisici dell'epoca erano a tutto tondo, sviluppavano la teoria e immaginavano, costruivano ed eseguivano esperimenti. Ad esempio Fermi con i raggi cosmici ad alta energia, che immaginò potessero esse generati da esplosioni di supernove.

        Dal 2010 invece lo studio della fisica del neutrino ha iniziato a differenziarsi, le sottoaree sono state assorbite dai gruppi di scienziati dell'ambito in cui viene osservato: a Catania sono divisi in nucleari, astroparticelle e acceleratori.

        Albergo cita sempre questi esperimenti e li chiede agli esami quindi li chiede anche lei per continuità!
    \subsection{Il neutrino nel modello standard}
        Il neutrino, come già detto, è parte delle particelle elementari che non sono da confondere con le particelle subatomice. Le particelle elementari sono i quark e i leptoni e per quanto ne sappiamo oggi son indivisibili. I protoni sono particelle subatomiche ma non sono elementari, essendo costituite da agglomerati di tre quark.

        Le interazioni vengono rappresentate attraverso altre particelle mediatrici come i fotoni per l'interazione elettromagnetica, i bosoni $W^\pm$, $Z^0$ per l'interazione debole, i gluoni per l'interazione forte e il bosone di Higgs che giustifica l'esistenza della massa.

        Ai neutrini viene attribuita massa nulla in maniera artificiosa, per far ``tornare i conti'' del modello standard, ma attraverso il metodo scientifico l'ipotesi è stata smentita attraverso la scoperta e lo studio delle oscillazioni. Lo studio delle oscillazioni è qualcosa che non viene previsto dal modello standarad quindi è \emph{beyond the standard model}. Matematicamente parlando, le oscillazioni sono interpretate come sovrapposizioni di autostati di massa distinti, e l'esistenza di massa prevede che il neutrino abbia massa, in contraddizione col Modello Standard. Al momento non abbiamo un numero per la massa ma solo dei limiti superiori, sappiamo che la massa del neutrino è non nulla e minore di una certa quantità. La massa del neutrino cambia come quella dei quark e degli altri leptoni cambiando famiglia, in ordine elettronico < muonico < tauonico.

        % La sua sezione d'urto è di \SI{10e-44}{\centi\meter\squared} e di conseguenza servono target estremamente massivi per poter avere un'interazione. Al gran sasso si usano rivelatori grandi come palazzi per poter osservare l'oscillazione del neutrino. km3net usa masse d'acqua enormi. L'esperimento Auger, di cui la prof Caruso fa parte, usa rivelatori vicino alle Ande e fa rivelazioni di fotoni e neutrini astroparticellari. Non essendoci grandi masse da sfruttare hanno rivelatori che tappezzano il terreno. In realtà misurano i neutrini che vengono dal lato opposto della terra che trapassano tutto il pianeta e interagiscono col terreno. Quindi anche se ancora non capiamo il neutrino, sappiamo che ci sono molti modi di rivelarli grazie agli avanzamenti della fisica delle particelle.
 
\section{Storia del neutrino}
        Carrellata di storia:
        \begin{enumerate}
            \item[1956:] Reines \& Cowan rivelazione della prima interazione diretta di (anti)-neutrino elettronico;
            \item[1962:] Danby et al prima rivelazione del neutrino muonico;
            \item[1975:] Perl scopre il leptone $\tau$;
            \item[2000:] La collaborazione DONUT del Fermilab registra il primo segnale proveniente da un neutrino tauonico.
        \end{enumerate}
        Ora torniamo indietro.
        \subsection{Il pomo della discordia}
            Pauli suppone che il neutrino abbia massa nulla, Fermi suppone che abbia una massa dell'ordine di grandezza del centesimo della massa dell'elettrone. Il modello standard attribuisce massa nulla. Dagli anni '40 agli anni '80 sono stati cercati diversi metodi per misurare la massa, che nel complesso si possono dividere in due grandi classi: le misure \emph{dirette} e le misure \emph{indirette}. A loro volta le misure dirette si dividono in filoni: studi cinematici in laboratorio---decadimenti, pochi ricercatori, faccio decadere cose e uso la meccanica classica e relativistica per studiare il moto del neutrino---, studio del doppio decadimento beta---senza neutrini, indicato dalla sigla $\betap\betap_0$---e infine neutrini da esplosione da supernovae.

            Le misure indirette invece si basano prevalentemente sullo studio delle oscillazioni del neutrino nella materia e sono metodologie più moderne. Quello che si va a misurare però non è la massa, ma piuttosto la differenza al quadrato $\Delta m^2$ degli autostati di massa dell'Hamiltoniana del neutrino secondo la teoria delle oscillazioni.

        \subsection{Misure dirette}
            \begin{enumerate}
                \item Fermi--Kurie plot: Fermi ipotizza che l'attribuzione di massa al neutrino influenzerebbe la parte finale---\emph{end-point}---dei grafici del decadimento beta. In questi grafici è rappresentato lo spettro di emissione in funzione dell'energia e con esperimenti molto precisi si osserva una curvatura nell'estremo destro dello spettro. Vai a vedere i grafici! Il succo è che l'andamento della curva sarebbe molto diverso se il neutrino non avesse massa, e basta una massa molto piccola a cambiare la forma dell'\emph{end-point}. Un problema consistente è che nell'\emph{end-point} del grafico cadono circa \num{10e-10} decadimenti del totale, che sono veramente pochi. Il limite superiore della massa ottenuto da questa misura è di \SI{10}{\eV}.
                
                Si usano decadimenti con un \Qv\ piccolo per evitare che il neutrino, che ha massa molto bassa, venga nascosto da un'energia di decadimento molto alta. Più è piccolo il \Qv meglio viene la misura. Serve inoltre un tempo di dimezzamento breve per avere molti decadimenti e poter fare più misure. Può infine essere utile avere una struttura nucleare semplice, dal momento che il decadimento è influenzato dall'interazione coulombiana e avere un nucleo piccolo e semplice introduce meno errori. Un ottimo nucleo che si presta a questo tipo di eperimento è il trizio che decade in elio 3 + elettrone + antinuetrino [mettere nella formula chimica]. Il trizio ha un tempo di dimezzamento di 12.3 anni.

                \item Esperimenti da decadimenti deboli: si sfruttano i decadimenti del pione in muone + neutrino muonico. oppure dal decadimento del tauone in tre pioni e un neutrino tauonico. Si sparano fasci di protoni su un target spaccando i nuclei del materiale e formando uno sciame di adroni e altre particelle secondarie, in modo simile agli sciami originati da raggi cosmici. Di conseguenza non li chiamo \emph{extensive air shower} ma solo \emph{shower}. La valanga di particelle infine attraversa dei rivelatori che sono in grado di risalire alla traiettoria del proiettile---rivelatori \emph{tracker}---e all'energia delle particelle---calorimetri formati da un mezzo denso e un mezzo scintillante, circondati da fotomoltiplicatori che convertono l'energia immessa nello scintillatore in segnale elettrico.
                
                Affinché questi esperimenti siano utili è necessario riconoscere tutte le interazioni che possono avvenire a valle. Comunque nello sciame secondario possono esserci pioni neutri o pioni carici, quelli neutri decadono in due gamma mentre quelli carichi decadono in coppie di muoni e neutrini. Quello che si fa è quindi generare sciami che contengano pioni e osservare i neutrini emessi dal suo decadimento. da qui massa del neutrino munoico minore di \SI{120}{\kilo\eV} %controllare!!

                \item Lo stesso può essere fatto osservando il decadimento dei $\tau$ che emettono tre pioni e un neutrino tauonico. [Vedi slide per dettagli sugli sperimentatori e l'anno]. I tauoni vengono generati nei collider presso il DORIS (1988) e il CERN (1995) trovndo una massa minore di \SI{24}{\mega\eV}
            \end{enumerate}
            Scala di massa? I limiti superiori ci fanno capire che le masse dei diversi sapori di neutrino sono di ordini di grandezza diversi.
            \begin{enumerate}
                \item Il doppio decadimento $\betap\betap_0$ può essere diverso in base al tipo di neutrino che può essere distinto in neutrino di Dirac o di Majorana. Goldhaber, Grozins e Sunyar osservano una reazione strana [vedi slide] dove il neutrino ha la stessa elicità di qualcosa boh?? di che sta parlando.? Ci sono neutrini destrorsi e sinistrorsi se immaginiamo l'elicità come una rotazione intorno alla direzione di propagazione. Ricordi l'esperimento di madame Wo dove c'era l'elicità dell'elettrone? Ecco, una cosa del genere.
                
                Pauli introduce uno spin di $1/2$ per bilanciare il momento angolare nel decadimento $\betap$. Possiamo dare l'elicità del neutrino come:
                \begin{equation*}
                    \eta = \frac{\vb{S}\cdot\vb{p}}{\norm{\vb{S}}\norm{\vb{p}}}
                \end{equation*}
                Se per un attimo immaginiamo lo spin come una rotazione della particella su sé stessa. Dall'eseprimento si osserva che l'elicità del neutrino è sempre \num{-1}, quindi viene detto particella \emph{sinistrorsa} o \emph{left-handed}. Se applichiamo il teorema di conservazione di carica, parità e tempo allora prevediamo che l'antineutrino possa essere \emph{destrorso} o \emph{right-handed}.

                Qualcosa sulla conservazione della coniugazione di carica? [rec a 38 minuti]

                
            \end{enumerate}bbbb
            % Distinguiamo il neutrino di Dirac da quello di Majorana in base alle idee proposte. Dirac propone che il neutrino e l'antineutrino sono distinti, il neutrino è sempre \emph{sinistrorso} e l'antineutrino è sempre \emph{destrors}: non possono esistere le configurazioni invertite, che vengono dette \emph{sterili}. Dirac afferma inoltre che si deve conservare il numero leptonico, quindi:
            % \begin{align*}
            %     \pi^+ &\rightarrow \mu^+ + \nu_\mu\\
            %     \pi^- &\rightarrow \mu^- + \overline{\nu}_\mu\\
            % \end{align*}
            % e inoltre
            % \begin{align*}
            %     \nu_\mu + N &\rightarrow \mu^- + \text{had.}\\
            %     \overline{\nu}_\mu + N &\rightarrow \mu^+ + \text{had.}
            % \end{align*}
            % [sentire rec]
            Il neutrino di Majorana invece è molto diverso dalla teoria di Dirac. Per lui il neutrino e l'antineutrino sono la stessa particella---segue banalmente che sono possibilie sia neutrini sinistrorsi che destroris---e sostiene che il numero leptonico non debba conservarsi:
            \begin{align*}
                \pionp &\rightarrow \muonp + \nu_\mu^-\\
                \pionm &\rightarrow \muonm + \nu_\mu^+
            \end{align*}
            essendo $\nu_\mu^+$ il destrorso e il $\nu_\mu^-$ il sinistrorso.

            Con i nostri esperimenti cerchiamo di capire se i neutrini sono come previsto da Dirac o come da Majorana. I fattori necessari per poter distinguere i due tipi di neutrini sono:
            \begin{enumerate}
                \item Una sorgente di neutrini come quelli che abbiamo visto fin'ora, insieme a rivelatori come calorimetri e rivelatori Cherenkov.
                \item La possibilità di invertire artificialmente l'elicità del neutrino nel corso dell'esperimento
                \item Studiare l'interazione del neutrino con elicità invertita: se il neutrino invertito non interagisce vuol dire che non esiste o è sterile come previsto da Dirac e viceversa se riesco ad osservarlo, secondo Majorana dovrebbe avvenire la
                \begin{equation}
                    \nu_\mu^+ + N \rightarrow \mu^+ + \text{had.}
                \end{equation}
            \end{enumerate}
            Il limite di questo programma è che fin'ora si è dimostrato impossibile invertire l'elicità del neutrino, per questo si ricercano i neutrini sterili nell'ambito della fisica astroparticellare o nei doppi decadimenti $\betap\betap_0$.