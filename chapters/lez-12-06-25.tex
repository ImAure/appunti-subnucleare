\section{boooh}
    riassunto lezione scorsa: avevamo visto le sorgenti di neutrini. poi nella lezione ha detto come si misura cosa, quali sono i vantaggi di alcuni metodi, cosa se ne trae etc. poi abbiamo visto lo stato dell'arte sulla ricerca sul neutrino (dirac/majorana, oscillazioni, fondo del neutrino, altro, roadmap fino al 2030, 2040, 2050). Dagli anni 60 è stato possibile indagare lo zoo delle particelle $qqq$, $q\overline{q}$. Kamiokande.... neutrini............. bip sono agglomerati di quark e altre particelle. per scoprire il neutrino è stato molto meno lineare che per scoprire le altre cose. Il modello standard ha avuto conferme su tutto tranne sul fatto che i neutrini hanno massa. Il modello standard non può essere sbagliato/inadatto/qualcosa, ha avuto troppe conferme per buttarlo via. IL MODELLO ASSEGNA ARTIFICIOSAMENTE MASSA NULLA AL NEUTRINO (frase da sapere a memoria). Sarebbe bello spiegare la massa del neutrino, la gravitazione, la materia oscura. Probabilmente facendolo non confuteremmmo il modello standard ma andrebbe corretto. Cambierebbe solo il suo campo di definizione (tipo la meccanica non relativstica a velocità bassa.) oltre certe situazioni vengono estese. Probabilmente il modello standard funziona in un certo range di energia, oltre un certo numero di \unit{\eV} non funziona più. Il fatto è che ora gli acceleratori ora permettono di raggiungere energie sempre più altre quindi ora forse abbiamo sforato il limite del modello standard. C'è una branca della fisica Teorica che tratta la fisica che estende il modello standard: Superstringhe e altre cose. sono teorie così estreme che sono circa impossibili da verificare per il semplice fatto che non abbiamo le tecnologie sufficienti a farlo. Si prova a usare le misure indirette come le increspature del fondo cosmic background. il fondo non è così isotropico quanto sembra quindi vedendo i gradienti di come è distribuito possiamo capire la fisica che va oltre. Alla lezione scorsa ha nominato una matrice 3x3 di cui sappiamo poche cose, solo 2 o 3 elementi sono noti in modo esatto. Un altro modo che abbiamo di ossservare fenomeni ad energie ancora più eleate sono quelle di usare i raggi cosmici che superano i \SI{e+17}{\eV}. Fisica delle energie estreme!

TORANDO A NOI

\chapter{Interazione Debole}
    Rivediamo alla luce di quanto fatto cosa è l'interazione debole mettendoci un po' di teoria, i diagrammi di Feynman e questa roba. La teoria è importante e va saputa.

    \section{La teoria}
        L'interazione debole è l'unica interazione attraverso cui il neutrino interagisce. È identificata dai decadimenti \betap. Non crea altre particelle ma regola i decadimenti. La si studia attraverso esperimenti estremamente difficili sia per i range che per le costanti di accoppiamento. La descrizione di interazione a 4 fermioni di Fermi è valida per energie del \unit{\mega\eV}.
        
        L'interazione debole non coinvolge solo leptoni ma anche i quark ed è \emph{l'unica} interazione attraverso la quale i quark possono cambiare sapore. I neutrini interagiscono solo debolmente mentre i quark sia fortemente che debolmente. L'interazione debole funziona grazie a tre propagatori dell'ordine del \unit{\giga\eV}. Vedi masse nelle slide $\Wp^\pm$: 80 GeV, Z 91 GeV. Si parla di interazione a corrente carica o a corrente neutra, in base a quale propagatore interviene: $\Wp^\pm$ o $\Zp^0$.

        I teorici Glashow, Weinberg e Salam, hanno sviluppato e pubblicato nel 1967 la teoria elettrodebole, in cui vengono unificate l'interazione elettromagnetica e l'interazione debole. Loro ipotizzano l'esistenza dei propagatori $\gamma$, $\Wp^\pm$, $\Zp^0$. La novità del loro modello, rispetto a quello della teoria debole, è quella di introdurre proprio lo $\Zp^0$ che è un mediatore neutro---come il fotone---ma dotato di massa. La teoria era così elegante e ben sviluppata che gli diedero il Nobel prima ancora di verificarla.

        L'esperimento che verificò l'esistenza del mediatore $\Zp^0$ è stato il Gargamelle nel 1973 (Pink Floyd reference). Cercava eventi del tipo:
        \begin{equation*}
            \ce{\neutrino_{\mu} + e^- -> \neutrino_{\mu} + e^-}\,.
        \end{equation*}
        che è possipile solo se è presente il $\Zp^0$. (inserire diagramma di feynman).

        L'esperimento UA1 (esiste anche UA2) che fece avere il Nobel a Carlo Rubbia misurò la massa del bosone $\Zp^0$ nel 1983. Si trattava del primo collisore protone--antiprotone, il $Sp\overline{p}S$. Dopo di lui fanno il LEP (Large Electron Positron collider) che comunque fa urtare un protone e un antiprotone nonostante il nome. La collisione avviene ad altissime energie, tanto che non si può trattare il fenomeno come un urto tra nuclei ma come un urto tra i singoli quark. Purtroppo non si può controllare l'energia di collisione tra i singoli quark. Il LEP ha una circonferenza di \SI{27}{\kilo\meter}, è stato operativo dal 1989 al 2000 e aveva quattro diversi rivelatori. Questo stesso collisore è stato utilizzato come \emph{fabbrica} di particelle. In una prima fase produceva milioni di $\Zp^0$ per ogni esperimento, in una seconda fase produceva $\Wp^\pm$ e altre cose.

        Tra i vari rivelatori nominiamo uno di riferimento, il DELPHI. È stato l'apparato più complesso fino all'arrivo del CMS al CERN. Riusciva a misurare in modo non distruttivo posizione e quantità di moto, per poi rivelare l'energia assorbendo le particelle.
        
        Riassunto: Gargamelle esistenza -> UA1 masse -> LEP altre cose
    \section{La scoperta della $\Zp^0$}
        Variando l'energia delle collisioni, il numero di eventi che si osservano al centro del rivelatorie cambia. Osservando il grafico si osserva una curva di risonanza molto evidente il cui picco corrisponde proprio alla produzione di $\Zp^0$. Avendo elevata precisione sull'energia delle collisioni, abbiamo un errore dello \SI{0.002}{\%} sulla massa del bosone $\Zp^0$.
        
        Le tracce sono diverse in base a ciò che viene prodotto durante la collisione. Il propagatore può decadere in diverse particelle che si comportano in modo diverso in base alla loro energia, alla loro massa e alla loro carica. (può decadere in una coppia leptone--antileptone o quark--antiquark). In 12 anni di misure al LEP abbiamo avuto solo conferme del modello standard (che comunque ha dei problemi). [TUTTO QUELLO CHE C'È nelle slide viene chiesto, almeno a ordini di grandezza, come tutte le sezioni d'urto. Questa slide la mettiamo alla fine per ricapitolare come tabella. nota: le vite medie non sono quelle dei propagatori ma proprio la durata media dell'interazione, vedi esempio del muone]
        
        L'interazione debole: 
        \begin{enumerate}
            \item non conserva: parità, coniugazione di carica, stranezza (vedremo cosa sono)
            In fisica delle particelle esistono simmetrie diverse da quelle spaziali, ovvero parità, coniugazione di carica e inversione temporale.

            La simmetria per coniugazione di carica si ha invertendo le cariche passando da particella a antiparticella.

            La simmetria per parità si ha quando cambio $\vb{r}$ con $-\vb{r}$.

            La simmetria per inversione temporale si ha per $t\rightarrow-t$. Ovvero posso ripercorrere un fenomeno all'indietro e ottenere lo stesso risultato?
            \item non dà origine a sistemi legat (Come agglomerati qqq che sono tenuti insieme dall'interazione forte.)
            \item Può avvenire attraverso processi
            \begin{enumerate}
                \item Leptonici tra leptoni
                \item semi-leptonici stra leptoni e quark
                \item non-leptonici, o adronici, tra soli quark.
            \end{enumerate}
            \item è un'interazione universale (vedremo che vuol dire): vuol dire che la costante di accoppiamento $g$ è la stessa per tutti i canali di decadimento, che non è una cosa così scontata, visto che i quark e i leptoni hanno natura molto diversa. I quark possono trasformarsi in altre cose e le loro transizioni sono dettate da frazioni diverse di $g^2$.? senti l'ultima mezz'ora della rec. Si mantiene l'universalità con l'interpretazione di Cabibbo rivista in chiave moderna. L'universalità esiste a meno di una correzione: l'interazione debole agisce sugli autostati di sapore dei quark (si pensava non agisse).
            \begin{equation*}
                \pqty{H_{em} + H_{s} + H_w} \ket{q} = E_{emsw}\ket{q}\,.
            \end{equation*}
            c'è di mezzo una rotazione, è nelle slides!!!!! l'angolo di rotazione per cambiare base viene detto angolo di Cabibbo $\theta_c$. Questa correzione permette l'estensione alla teoria forte usando la matrice $3\times3$ di Cabibbo--Kobayashi--Maskawa.nooooooooooooo la polizia nooooo
            \item tutti i tipi di interazioni deboli sono caratterizzate dalla stessa costante di accoppiamento.
        \end{enumerate}
        Vedi diagrammi di Feynman!
        
    \section{Richiami di processi tra particelle}
        Ricordiamo che la sezione d'urto è coinvolta nel numero di conteggi per unità di tempo. Dobbiamo ricordare che esiste la matrice di transizione $M$ il cui modulo quadro di un elemento di matrice $\abs{\tensor{M}{_{ij}}}$dà la probabilità di transizione tra l'$i$-esimo e il $j$-esimo stato. La seconda regola d'oro di Fermi dà la probabilità per unità di tempo della reazione etc vedi slide. C'entra la densità dello spazio delle fasi. Bisogna usare trasformazioni di Lorentz e indeterminazione di Heisenberg.

        \subsection{Diagrammi di Feynman}
            Le linee intere sono le particelle reali, le linee tratteggiate sono le particelle virtuali. è un schematizzazione spazio-temporale, di solito con il tempo da sinistra verso delstra. ci sono delle regole!!
             
            Sono incredibilmente efficaci, sono semplici ma densi di informazioni e a forte impatto visivo: funzionano. Ma come si realizzano?

            Dalla meccanica quantistica sappiamo che possiamo calcolare con esattezza la probabilità degli eventi calcolando l'ampiezza delle transizioni, numeri complessi il cui modulo quadro è proprio la probabilità. La sezione d'urto si ricava da queste cose. Poi si usano le approssimazioni attraverso il metodo perturbativo. I diagrammi di Feynman danno proprio i termini perturbativi.

            Le componenti del diagramma di Feynman sono: Vertici, particelle esterne, propagatori. In ogni vertice si conserva il $4$-impulso. Le frecce indicano la direzione di una particella o la direzione opposta per un'anti particella.

            Le particelle non scelgono quale diagramma seguire. Ogni evento è di fatto una combinazione lineare di ogni possibile decadimento.

            Esempio: decadimento del muone.

            I muoni si usano per la tomografia!! Per fare le radiografie alle barche al porto o vedere i buchi nelle montagne o nelle piramidi.

            Si può calcolare la sezione d'urto, facoltativo. se c'è qualcosa che ci piace e glielo vogliamo portare e lo vogliamo approfondire va bene.


    \section{Rotture di simmetria}
        Il passaggio da $\vb{r}$ a $-\vb{r}$ è una riflessione puntuale rispetto all'origine. Se facendo questo il fenomeno cambia abbiamo informazioni sulla chiralità della particella e del fenomeno: l'interazione debole è chirale.
        idk vedi slide è disordinatissimo.        

        La coniugazione di carica commuta le particelle con le altre particelle e cambia le loro cariche. non solo quella elettrica ma anche la carica di colore dei quark. Se le leggi fisiche sono invarianti per trasformazioni di coniugazione di caricha allora abbiamo simmetria per coniugazione di carica. L'interazione debole non resta invariata, mentre gravità, elettromagnetismo e interazione forte no.

        slide che non si capisce con un elenco di cose. i neutrini hanno chiralità definita. questo è certo.ògvu

        Importante elicità e chiralità del neutrino che è la proiezione dello spin $\vb{S}$ sul vettore quantità di moto.

        Quando si parla di interazione debole si rompe anche la simmetria CP, ovvero se flippo coniugazione di carica e allo stesso tempo inverto le coordinate spaziali. Quindi materia e antimateria si comportano in modo diverso. Il protone si comporta diversamente dall'antiprotone. È stato scoperto con il Kaone nell'esperimento Kronin 1964. In base al suo stato CP decade in modi diversi.

        Forse la lieve asimmetria nelle leggi della fisica ha permesso l'esistenza della materia. Vedi condizioni necessarie di Andrej Sacharov.

        S A|S A
        L E|L E

        Serve altra fisica Beyond the Standard Model. Basta gol donno
        Teo rema CPT è importante. Permette alle statistiche quantistiche di essere come sono (Fermi--Dirac con spin semintero, Bose--Einstein con spin intero).Grazie per la pazienza
        
+
