\chapter{Storia della Fisica delle Particelle}
    \section{Il concetto di interazione}
        Fin dall'antichità l'uomo ha cercato di comprendere la natura e le forze che la governano. Le interazioni tra gli oggetti sono state osservate e studiate, portando alla formulazione di leggi fisiche che descrivono questi fenomeni. In un primo tempo l'interazione è stata studaita come un fenomeno macroscopico che prende il nome di \emph{forza}, simile alla parola \emph{sforzo}: per alterare lo stato di un sistema---come ad esempio cambiare la velocità di un oggetto o deformarlo---è necessario compiere un qualche tipo di sforzo. Quando però cerchiamo di descrivere fenomeni più complessi, come quelli che avvengono a livello atomico o subatomico, dove diventa difficile stabilire in quale momento avvenga il contatto, il concetto di forza diventa insufficiente. In questi casi, è più utile parlare di \emph{interazione}.
        
        In fisica delle particelle, il concetto di interazione assume un ruolo primario dal momento che gran parte del lavoro dei fisici è dedicato allo studio della natura di queste interazioni e al tentativo di unificarle tutte tra loro, ovvero trovare un'unica forza di cui tutti i fenomeni fisici siano manifestazioni. Un primo esempio di unificazione è stato quello di Sir Isaac Newton, che con la sua Teoria della Gravitazione Universale ha dimostrato che la forza di gravità che agisce tra due corpi celesti, come la Terra, la Luna e il Sole, è la stessa che fa cadere gli oggetti sulla superficie terrestre. Un altro esempio è la Teoria Elettromagnetica di James Clerk Maxwell, che ha unificato le forze elettriche e magnetiche in un'unica forza elettromagnetica. Sebbene queste due teorie siano state migliorate e generalizzate nella Relatività Generale di Albert Einstein e nell'Elettrodinamica Quantistica di Dirac, Pauli, Fermi, Feynman e molti altri, esse non sono sufficienti a spiegare tutti i fenomeni fisici conosciuti, specialmente quelli che avvengono a scale molto piccole.

        Le interazioni che sono state teorizzate per descrivere la fisica dei nuclei atomici---come il decadimento $\beta$ e l'interazione del neutrino---sono l'interazione nucleare debole, che è stata successivamente unificata con l'interazione elettromagnetica nella Teoria Elettrodebole, e l'interazione nucleare forte, che attraverso la Cromodinamica Quantistica spiega le interazioni tra \emph{quark} e \emph{gluoni}. Queste due interazioni sono raccolte in modo sistematico nella Teorai del Modello Standard delle Particelle Elementari ma non sono tuttavia unificate tra loro, né con la gravità. Sono inoltre presenti delle inconsistenze legate all'attribuzione artificiosa di massa nulla al neutrino, che invece mostra dei comportamenti spiegabili solo da una massa non nulla.

        \begin{table}
            \centering
            \begin{tabular}{lccr}
    \toprule
    \textbf{Interazione} & \textbf{Portata} & \textbf{Intensità relativa} &\textbf{Portatore} \\
    \midrule
    Gravitazionale & $\infty$ & \num{1} & Gravitone* \\
    Elettromagnetica & $\infty$ & \num{e+36} & Fotone  \\
    Nucleare debole & \SI{e-18}{\meter} & \num{e+33} & Bosoni $W^\pm$, $Z^0$  \\
    Nucleare forte & \SI{e-15}{\meter} & \num{e+38} & Gluone \\
    \bottomrule
\end{tabular}
            \caption{Le quattro interazioni fondamentali della natura. *Il gravitone è una particella ipotetica, non ancora osservata, che dovrebbe mediare l'interazione gravitazionale.}
            \label{tab:interazioni-fondamentali}
        \end{table}
        Le principali interazioni descritte fin'ora prendono il nome di \emph{interazioni fondamentali} e sono riportate in tabella \ref{tab:interazioni-fondamentali}.

    \section{Classificazione delle particelle}
        \begin{figure}
            \centering
            % SM PARTICLES plus TENSOR (like Wiki)
{\sffamily\small
\foreach \f in {0}{ % fill particle boxes
\foreach \opQua/\opLep/\opNu/\opGlu/\opGam/\opWeak/\opHig/\setGen in {%
  % highlight different groups of particles,
  % by reducing the opacity of others
  1/1/1/1/1/1/1/1          % highlight everything
}{ % loop over opacities
\begin{tikzpicture}[fill box=\f]
  \message{^^JSM particles: fill box=\f}
  
  % HEADERS
  \def\nbosoncols{3} % number of boson columns
  \headers
  
  % QUARKS
  \pic (QU) at (1,4) {
    particle={quarkcol}{u}{up}
  };
  \pic (QC) at (2,4) {
    particle={quarkcol}{c}{charm}
  };
  \pic (QT) at (3,4) {
    particle={quarkcol}{t}{top}
  };
  \pic (QD) at (1,3) {
    particle={quarkcol}{d}{down}
  };
  \pic (QS) at (2,3) {
    particle={quarkcol}{s}{strange}
  };
  \pic (QB) at (3,3) {
    particle={quarkcol}{b}{bottom}
  };
  \node[quarkcol,bflabel,above right=0pt and -2pt]
    at (QD-sw) {QUARKS};
  
  % LEPTONS
  \pic (EL) at (1,2) { %[pshift={(50:0.9)}]
    particle={leptoncol}{e}{electron}
  };
  \pic[symb shift=(-90:0.6pt)] (MU) at (2,2) {
    particle={leptoncol}{$\PGm$}{muon}
  };
  \pic (TAU) at (3,2) {
    particle={leptoncol}{$\PGt$}{tau}
  };
  \pic[scale name=0.83] (NE) at (1,1) {
    particle={leptoncol}{$\PGn{\text{e}}$}{electron\\[-3pt]neutrino}
  };
  \pic[symb shift=(-90:0.6pt),scale name=0.83] (NM) at (2,1) {
    particle={leptoncol}{$\PGn{\PGm}$}{muon\\[-3pt]neutrino}
  };
  \pic[scale name=0.83] (NT) at (3,1) {
    particle={leptoncol}{$\PGn{\PGt}$}{tau\\[-3pt]neutrino}
  };
  \node[leptoncol,bflabel,above right=0pt and -2pt]
    at (NE-sw) {LEPTONS};
  
  % GAUGE BOSONS
  \pic (GLU) at (4,4) {
    particle={gaugecol}{g}{gluon}
  };
  \pic (GAM) at (4,3) {
    particle={gaugecol}{$\gamma$}{photon}
  };
  \pic (W) at (4,2) {
    particle={gaugecol}{W}{W boson} %$\mathrm{W}^\pm$
  };
  \pic (Z) at (4,1) {
    particle={gaugecol}{Z}{Z boson} %$\mathrm{Z}^0$
  };
  %%%\pic[scale name=0.7] (L) at (5.6,1) {
  %%%  particle={gaugecol}{LQ}{leptoquark}{% %^0$
  %%%    ?}{?}{0 or 1} %>1\TeV
  %%%};
  \node[gaugecol,bflabel,below right=0pt and 2pt]
    (GB) at (Z-se) {GAUGE BOSONS};
  \node[gaugecol,bflabel,below right=-1pt and 2pt,scale=0.7]
    at (GB.south west) {VECTOR BOSONS};
  
  % SCALAR BOSONS
  \pic (HIG) at (5,4) {
    particle={scalarcol}{H}{Higgs}
  };
  \node[scalarcol,bflabel,above left=-2pt and 2pt]
    at (HIG-se) {SCALAR BOSONS};
  
  % TENSOR BOSONS
  \pic (GRA) at (6,4) {
    particle={tensorcol}{G}{graviton}
  };
  \node[tensorcol,bflabel,above left=-2pt and 2pt]
    (TB) at (GRA-se) {TENSOR BOSONS};
  \node[tensorcol,bflabel,above left=-1pt and 2pt,scale=0.7]
    at (TB.north east) {HYPOTHETICAL};

   % INTERACTION GROUPS
  \ifnum\pgfkeysvalueof{/tikz/fill box}=0
  \begin{pgfonlayer}{back} % draw on back
    
    % STRONG INTERACTIONS
    \def\R{11.5pt}
    \fill[intgroup,opacity=0.5*\opGlu] %=blue!20!white]
      (QU-p)++(0,\R) -- ($(GLU-p)+(0,\R)$) arc(90:-90:\R)
      to[out=-180,in=90,looseness=1.2] ($(QB-p)+(\R,0)$) arc(0:-90:\R)
      -- ($(QD-p)+(0,-\R)$) arc(-90:-180:\R)
      -- ($(QU-p)+(-\R,0)$) arc(180:90:\R)
      -- cycle;
    
    % ELECTROMAGNETIC INTERACTIONS
    \def\R{13.5pt}
    \fill[intgroup,opacity=0.5*\opGam] %=green!20!white]
      (QU-p)++(0,\R) -- ($(QT-p)+(0,\R)$) arc(90:0:\R)
      to[out=-90,in=180,looseness=1.2] ($(GAM-p)+(0,\R)$) arc(90:-90:\R)
      to[out=-180,in=90,looseness=1.2] ($(TAU-p)+(\R,0)$) arc(0:-90:\R)
      -- ($(EL-p)+(0,-\R)$) arc(-90:-180:\R)
      -- ($(QU-p)+(-\R,0)$) arc(180:90:\R)
      -- cycle;
    
    % WEAK INTERACTIONS
    \def\R{15.5pt}
    \fill[intgroup,opacity=0.5*\opWeak] %=mypurple!20!white]
      (QU-p)++(0,\R) -- ($(QT-p)+(0,\R)$) arc(90:0:\R)
      -- ($(QB-p)+(\R,0)$)
      to[out=-90,in=180,looseness=1.4] ($(W-p)+(0,\R)$) arc(90:0:\R)
      -- ($(Z-p)+(\R,0)$) arc(0:-90:\R)
      -- ($(NE-p)+(0,-\R)$) arc(-90:-180:\R)
      -- ($(QU-p)+(-\R,0)$) arc(180:90:\R)
      -- cycle;
    
  \end{pgfonlayer}
  \fi
\end{tikzpicture}
}
} % close foreach loop over \f
}
        \end{figure}
    \section{Storia delle scoperte}