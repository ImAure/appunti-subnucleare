%====================%
%                    %
%   29 MAGGIO 2025   %
%                    %
%====================%

\section{29 maggio 2025}
Torniamo alla questione neutrino di Dirac e neutrino di Majorana.
\subsection{esperimento concettuale e doppio decadimento \betap}.
            Il decadimento \betap\ è un decadimento con emissione di un neutrino, SE ESISTONO, i doppi decadimenti \betap\ dovrebbero emettere due neutrini(?) si scrive come $\betap\betap_0$ perché nel complesso non viene emesso il neutrino
            \begin{equation}
                \pqty{A,Z} -> \pqty{A,Z+2} + e^- + e^-
            \end{equation}
            Se un nucleo decade \betap\ emette un elettrone e un neutrino. Un doppio \betap\ permesso dovrebbe essere un decadimento con 2 elettroni e 2 neutrini. Ma può esistere un doppio \betap\ dove non viene emesso nessun neutrino? o meglio dove un neutrino viene emesso e riassorbito e restano solo i due elettroni. Questo tipo di decadimento sarebbe possibile se l'antineutrino coincidesse con il neutrino stesso e avesse massa non nulla.

            In che modo permette di distinguere tra Dirac e Majorana? Il doppio decaimento senza neutrino non è possibile per i neutrini di Dirac ma solo per quelli di Majorana. In ogni caso è un processo del secondo ordine, ovvero sfavorito probabilisticamente e che avviene solo quando un decadimento del primo ordine come il classico \betap\ è proibito energeticamente.

            Nel caso di Majorana, siccome non si osserva il neutrino, parliamo di neutrino virtuale che viene scambiato (tipo i fotoni virtuali dell'interazione EM?). Ora noi abbiamo appurato che il neutrino ha massa non nulla.

            Kamiokande e osservatorio del gran sasso (MACRO, esperimento completamente diverso! quindi conferma molto bene) avevano scoperto le oscillazioni del neutrino che sono possibili solo se la massa è non nulla.
            
            La reazione divisa in due step sarebbe:
            \begin{align}
                \pqty{A,Z} -> \pqty{A,Z+1} + e^- + \neutrino \\
                \neutrino\pqty{h = -1} + \pqty{A,Z+1} -> e^- + \pqty{A,Z+2}
            \end{align}
            questo processo richiede uno spin fotomoltiplicatori
            Sono importanti dei punti fondamentali: l'ampiezza di probabilità per produrre un neutrino denna lprima transizione è pari a $m_\neutrino / E_\neutrino$, quindi quella per il doppio decadimento è proporzionale al quadrato di $m_\neutrino$. Chiaramente se la massa del neutrino è nulla la probabilità di osservarlo
            
            che esperimenti sono stati fatti su questa cosa?
            \begin{equation*}
                \ce{_{32} Ge ^{76} -> _{34}Se^{76} + e^- + e^-}
            \end{equation*}
            c'è un esperimento che si chiama CUORE fatto da Fiorini e co. (in corso, è l'esperimento di riferimento sul $\betap\betap_0$) sottoterra underground in silenzio cosmico (quindi bloccando tutta la radiazione cosmica), le condizioni devono essere super controllate e studia dei decadimenti rarissimi. L'apparato è grande quanto un palazzo di 4 piani e prima è stata testata larealizzabilità dell'esperimento con un prototipo chiamato CUORICINO. Servono fisici delle particelle e delle astroparticelle e i fenomeni vengono studiati a temparture prossime a \SI{0}{\kelvin}. Ma a temperature basse succedono cose strane! già i sipm a -40 gradi celsius e se magari il fotosensore resiste grazie a delle celle climatiche ma la colla e tutta la parte strutturale e l'elettronica e cavi cedono a basse temperature! Se vado sotto figurati!

            Comunque Fiorini nel primo esperimento (non cuore) faceva decadere germanio in selenio e non leggeva nessun segnale. L'esperimento MACRO era per trovare monopoli magnetici e il fatto di non averli trovati permette di dare un limite superiore a quanti se ne possono vedere (sicuramente meno di quelli che lo strumento può rivelare, anche se dovessero essere 0). Quindi sostanzialmente diedero un limite superiore alla vita media del germanio e alla massa del neutrino. Maier (1994) e Heidelberg--Mosca (1998) danno una nuova stima migliorata.

            In seguito l'esperimento NEMO-3 nei laboratori sotterranei del Frejus (2003) dà risultati molto migliori (vedi slide per i numeri).

            Anche il NEMO è estremamemtne complicato (ci sono delle belle foto!). Da questi esperimenti (che sono misure dirette) possiamo iniziare a tirare giù delle tabelle su caratteristiche e limiti superiori.

        \subsection{Aspe}
            gli esperimenti da misure indirette danno stime migliori sulla massa del neutrino, poi le facciamo!
        \subsection{Sorgenti}
            Sorgenti! Naturali:
            \begin{enumerate}
                \item neutrini terrestri da radioattività ambientali
                \item neutrini atmosferici da raggi cosmici
                \item neutrini solari (catene pp, CNO)
                \item neutrini da esplosione di supernovae
                \item neutrini dal Big Bang 
                \item neutrini da interazioni dei raggi cosmici primari tra loro nello spazio intergalattico
            \end{enumerate}
            Di quelle artificiali abbiamo:
            \begin{enumerate}
                \item neutrini ad reattore
                \item neutrini da acceleratore
            \end{enumerate}
            Non si possono ovviamente fare esperimenti in grado di rivelare tutti i tipi di neutrini a tutte le possibili energie, per questo ogni ``tipologia'' di neutrino deve essere studiata con strumenti specifici. Quello che si può fare in seguito è combinare i dati nei grafici di esclusione e estrapolare più informazioni possibili. Vediamo ora le sorgenti di origine naturale:
            \begin{enumerate}
                \item[\textbf{Terrestri.}] Al gran sasso rivelatore borexino bellissimo! foto belle! fa neutrini solari, fa la stessa cosa del kamiokande ma consapevolmente (sa che ne deve beccare un terzo). Questo rivelatore becca anche neutrini terrestri dal decadimento di elementi come Uranio e Torio. Sono rivelati dal superkamiokande e dal borexino. I modelli attuali sulla struttura interna della terra sono solo teorici. Juno forse aiuterà: Un altro grande osservatorio che fa questa cosa à JUNO, stessa cosa, una palla di acqua con fotomoltiplicatori. è in una caverna in cina circondato da una decina di reattori nucleari e cerca di rivelare proprio quelli, oltre che quelli atmosferici, da supernove e dal core terrestre. Ci sono stati 10 anni per pensarlo e metterlo a punto, qualche mese fa è stata riempita la sfera di scintillatore liquido e ora inizierà la presa dei dati. Perché il neutrino è un \emph{messaggero del cosmo}? è una sonda potentissima ma che significa????? beh!!!! ci porta informazioni dal core della terra e ai geofisici torna molto utile averlo come strumento. è meglio di fare carotaggi, non possiamo mica andare sottoterra.  
                
                \item[\textbf{Atmosferici.}] Per quel che riguarda i neutrini atmosferici, sono quelli prodotti durante le interazioni che avvengono nelle \textit{Extensive Air Shower} (\SI{e+21}{\eV}). Sono generate da raggi cosmici e altre particelle cariche ad alte energie sparate dal Sole. Se ne formamo molti di tipo muonico e tauonico e contninuano a formarsene finché lo sciame riesce a produrre nuove particelle. Alcune particelle arrivano a terra e vengono dette penetranti. Alcuni muoni viaggiano per 300 metri sottoterra ma i neutrini molto di più! questi esperimenti permettono di misurare sezioni d'urto, energiem oscillazioni etc. sta ripetendo le stesse cose sul fatto che oscillano e ne vediamo di meno. quelli atmosferici sono utili a studiare le proprietà dei neutrini per poi rivelare meglio quelli che vengono da altre sorgenti.
                
                \item[\textbf{Solari.}] I neutrini solari sono proposti da Bethe nel 1938 con un modello di reazioni nucleari nel sole. Il ciclo pp unisce due protonie e con un deacdimento \betap\ si produce un deuterio etc... Ci hanno permesso di studiare le reazioni nucleari che avvengon oall'interno del sole. Anche qui da catene pp e CNO (molti meno) ci aspettiamo tot neutrini dal modello stadard del sole di (bacoll??) e se ne trovavano sempre di meno (vedi kamiokande). ovviamente i fisici solari credevano al modello standard del sole mentre i particellari credevano alle loro misure. Mancavano perché i neutrini elettronici attraversando il sole estremamente denso si trasformano e diventano muonici e tauonici. Quindi giusto che ne vediamo di meno! tra l'altro nella materia si innesca un fenomeno di risonanza e i neutrini oscillano maggiormente nella materia.
            
                Vedere gli schemini, il Borexino ha dato misure precisissime sulla catena pp ma anche per i decadimenti di Berillio e altre reazioni e decadimenti.
                
                Che problemi abbiamo coi neutrini solari? Il Sole produce un flusso di \SI{6e+10}{\per\second\per\centi\meter\squared} neutrini, veramente tanti! Questo flusso è stato misurato in diversi contesti e con diversi energie, e tutti gli esperimenti misuravano più o meno la metà dei neutrini attesi. Stiamo parlando degli anni 60, ci sono voluti 40 anni per scoprire che erano le oscillazioni. Il modello solare di bacoll è confermato!
                
                \item[\textbf{Da Supernovae.}] Le stelle \textbf{SCHIATTANO!!! PAMMMMM} producono un sacco di neutrini con delle caratteristiche specifiche. Sono VERAMENTE VERAMENTE energetici. CI permettono di studiare le stesse supernove e ci sono molti astrofisici che studiano le supernove quindi è una sorgente più o meno conosciuti. C'è l'esperimento LWD che è stato costruito dopo la famosa supernova nel 1987-A. Purtroppo da noi ce n'è una ogni 30 anni, magari a preve ne scoppia una! Speriamo in JUNO e in LWD.
                
                In seguito all'esplosione del 1987 nella nube di Magellano sono stati  rivelati 11 neutrini in 10 secondi da due diversi rivelatori che cercavano neutrini solari. Questi neutrini venivano dalla povera stella scoppiata (R.I.P. sellina) ma questo confermava il modello teorico delle Supernovae.

                A noi interessa in particolare capire in quale momento del collasso stellare avviene l'emissione dei neutrini da Supernovae, si nota che arriva un burst di neutrini e posso misurare con quanto ritardo arrivano rispetto alla luce. Da questo ritardo posso ricavare direttamente la massa, si è ottenuto un limite superiore di \SI{12}{\eV}. Gli altri metodi non riescono a ottenere risultati più precisi, ma non possiamo aspettare 30 anni ogni volta!
                % +
                % m

                \item[\textbf{Cosmogenici.}] Sono i neutrini originati dall'interazione tra raggi cosmici e altra materia intergalattica---non l'interazione con l'atmosfera---e si tratta dei neutrini a più elevate energie conosciute. È possibile che provengano dai nuclei galattici attivi (AGN). Sono stati rivelati da Ice Cube e km3net, famosi!
                
                \item[\textbf{Cosmologici.}] Si ipotizza l'esistenza di un fondo di neutrini primordiali dal big bang, sarebbero come il fondo cosmico a microonde, che dovrebbero permeare l'universo. Li cerchiamo ma non abbiamo conferma, servono tecniche molto raffinate ed esperimenti molto diversi da quelli fatti fin'ora. è un campo aperto ma molto recente.
                 
                Secondo le ipotesi sarebbero stati prodotti moltissimi neutrini durante il Big Bang e dovrebbe essere rimasto la \emph{radiazione fossile di neutrini} di neutrini e antineutrini. Questi neutrini dovrebbero aver perso moltissima della loro energia nell'espansione dell'universo, giungendo a \SI{e-9}{\eV}, e sono per questo difficilissimi da notare nonostante siano distribuiti in modo isotropo con una densità $n \sim \SI{300}{\per\centi\meter\squared}$.
            \end{enumerate}
            È di grande utilità, soprattutto per la comprensione delle oscillazioni, lo studio dei neutrini da sorgenti artificiali:
            \begin{enumerate}
                \item[\textbf{Dal nucleare.}] I reattori nucleari vengono usati come sorgenti. Ci si mette a 10--20 metri fino a max 1km, per questo sono di tipo \textit{short baseline}. Serve conoscere bene cosa il reattore produce per poter prevedere il comportamento del neutrino lungo il tragitto. Si va a vedere il flusso a valle e lo si confronta col flusso a monte per avere informazioni sulle oscillazioni e sulla massa.
                
                Nei reattori nucleari l'uranio 235 decade alla grande e si autoalimenta producendo energia e causando altri decadimenti. I nuclei figli dell'uranio, Kripton e Bario, hanno troppi neutroni per sopravvieve e decadono \betap. Il risultato sono neutrini. Un reattore che produce 3 GW di energia produce 6 per 10 alla 20 neutrini al secondo su tutto l'angolo solido.

                L'osservatorio JUNO ha anche delel appendici fatte apposta per queste misure. Ad esempio TAU che è una palla di SiPM di qualche decimetro di diametro e misura i neutrini dai reattori nucleari.
                
                \item[\textbf{Da acceleratori.}] Più o meno la stessa cosa si fa con gli acceleratori. Qui mettiamo rivelatori da 100km a 1000km di distanza e ricerchiamo allo stesso modo i fenomeni di oscillazioni.
                
                Funziona accelerando protoni---puri e guidati come necessario---che poi producono altre particelle come kaoni e pioni. Metto un materiale assorbitore denso come il piombo che arresti i muoni e restano i neutrini che attraversano distanze lunghe. Ad esempio posso produrre nuetrini al CERN e misurarli al Gran Sasso con l'esperimento Opera che ha verificato l'esistenza del neutrino tau e delle oscillazioni.

                Questi esperimenti sono chiaramente long-baseline
            \end{enumerate}
                
    \section{Come si misurano i neutrini}






    Alla fine della presentazione di questo slides, vi rende l'erop una decisione complessata. Sono sattati a a a a a a a a a a a a a a questo c'è la strada a cui il digliore oggi definiamo, dopo il decalimento beta, dopo il decalimento beta già la frase mi fa capire se è un decalimento beta, è un'emissione con due neutrini, se sia possibile o meno questo è tutto da sapere, anche vero che ancora oggi sono riconso una serie di esperimenti, intanto per velocità di esistenza di questo tipo di decalmenti che non sono fabbili, da un'unitivista probativistico della realizzazione di decalmenti o transizioni o integrazione, da un'unitivista lucriare o attriciare. Il 15.0 è stato decale dopo il decalimento beta senza emissioni, in realtà è un'emissione di rassorbimento, alla fine della fiera, della fase secondaria della transizione stessa, della delitrazione stessa, in realtà i neutrini non esistono e non possono essere vera, ma abbiamo lo bello attraverso questo schema che credo sia un bordo discritto, sufficientemente discritto, sì. Allora, un dottore del decalimento beta senza neun clima, senza neun clima, in dire appunto rappresentato con lo zero a te, è un decalimento in cui scelgo una sostanza, un elemento chimico di numero di massaio e numero atomico z, che possa decadere in un elemento chimico con numero atomico maggiorato di due unità, il numero di massaio e numero numero z più due, l'emissione di due elettroni, non una elettrone e il sinneurino. Fondamentalmente sto immaginando qualcosa di pieno, se rappresentiamo questa schema di stazione elettronica, se oggi finalmente funziona anche il mouse e il pulsore, se immaginiamo il decalimento z, qui abbiamo l'acqua con il suo nucleo, con i suoi tronini di di di tronini di di di di di di di di di di di di di di di tronini tronini tronini di tronini di tronini di tronini di tronini di tronini di tronini di di di di di tronini tronini tronini tronini tronini tronini tronini di tronini di tronini di tronini di tronini di tronini di di di tronini di tronini di tronini di di di di di di di di di di di di di di di tronini di di di di di di di di di di di di di tronini di tronini di di di di di di di di di di di di di di di di di di di di di di di di di di di di di di di di di di di di di di di di di di di di di di una una di rappresentazione delle interazioni che vedremo nelle prissioni successive che avrete ancora meglio l'utilizzo di questo simbolo, ovvero quest'arco di circunferenza a rappresentare sostanzialmente un'emissione, ma anche un annullamento dei rioclinistessi e di fatto nello stato finale effettivo, reale troviamo soltanto due elettroni questo tipo di decalmenti in beta-beta a zero sono possibili in linea di principio si potizza qualora l'altidiotino elettronico coincidesse con il diutino stesso e con una massa del niotrino diversa da zero abbiamo visto quali sono gli elementi fondamentali da un punto di vista concettuale per realizzare un esperimento che ci consente ad un stimolo di mani orari e metri di di altre nell'andare in questa direzione e comprendendo quella che la raccundano di questi decalmenti l'idea è poi in utilizzare sostanze specifiche, debbore avere delle particolari cartelisti per capire quali è è nome di seguito, e poi realizzare questo tipo di decalmento in l'algoratorio e andare ad efficare il stato finale l'esistenza o l'assenza del niotrino questo processo è il primo distingue tra le otrimiti d'acchina mai orama perché il doppio determinato beta beta al zero è impossibile per le otrimiti d'acchina perché viola il numero elettronico, sostanzialmente io non ho lo stesso numero di elettroni nel stato iniziale che è il stato finale, ma è permesso se i neptronini fossero di maggiorana ovvero particella di un incidente con la propria antiparticella e una massa diversa da zero in ogni caso sia con emissione con l'assenza di un imprimi dello stato finale è un processo del secondo ordine, ovvero probabilisticamente spavolito è possibile quando il decalmento beta, il primo ordine e questo avviene per alcune sostanze chimiche è energeticamente produttito quando lo vediamo meglio, questo è un'immagine editoria, quindi sto immaginando intanto capiamo che nel decalmento, in una sostanza di tipo AZ più 2 via un'etna di energia e quello che io posso misurare nello stato finale è la somma delle energie degli elettroni messi, in caso di assenziato di primo stesso vediamo meglio questa slide dove qui lo rappresentiamo in maniera non pictorica, diciamo maschematica e dove sostanzialmente stiamo immaginando neutrone e protone come costituiti da particelle elementali, quindi work a peranna la transazione avviene e il decalmento beta, il neutrone si trasforma in un protone che mette un elettrone tramite l'introfagiatore Dr. B e quello che accade è che è un processo a due step con elizione e riassorbimento di un neutro inolviguale quello che abbiamo rappresentato pictoricamente in questo modo è quello più fisico delle particelle, si dice una elizione e riassorbimento di un neutro inolviguale perché ipotiziamo che avvenga questo tipo di processo, ma di fatto nel stato finale non troviamo le autitune e di conseguenza l'esperimento che conferma la non-existence neutrige nel stato finale ci direbbe che questo tipo di processo si è effettivamente verificato e questo tipo di processo si potrebbe verificare perché avrò ancora un punto di domanda molto importante e si potrebbe verificare solo se in depredasse il neutro di Majorana vorrei farvi anche notare e osservare questo un po' basso appunto alla paio di domande che sono state posti a finizione nell'altro ieri che a questo punto appunto un un ma pare che sia effettivamente il vostro collega mi chiedeva se oggi sappiamo distingue tra il neutro di Majorana e di Atta la risposta è no se avvengono esperimenti, se sono ricorso esperimenti che cercano che mirano a rispondere a questa domanda è la risposta sì ma si chiedeva anche se ci fosse effettivamente una teoria strutturata, organica di questo neutro di Majorana e dell'esistenza di i carimenti doppio per petà zero poi non lo verifichiamo è una di quelle risponde le quali doveva ancora rispondere da un puro di istacentrico, ma se ci fosse una teoria adeguando la teoria è quella proprio di Ettole Majorana Ettole Majorana non solo il poco di zoe esistenza di questo neutrino con queste particolari relativistiche ma sviluppo Caccholi, Scrisse Ettole Majorana fu riconosciuto dei ricoperni effettivamente sono studenti più geniare, stiamo parlando di una classe di studenti di discegoli e ricoperni il gesto uno dei quali da più rossi da mani, di addocchialini, ecc. ma colleghi o giovani collaboratori hanno scritto la fisica delle particelle delle astro-particelle, nel secolo del secolo sconsolso quindi hanno tutti i gradi di competenza, di intelligenza, di preparazione di questi fisici e avrete outstanding, come si diceva, un'idea si è fuori, da un ordinario e pure in questo gruppo, i ragazzi di Diata Lusperna, c'era chi è emergeno con la di più che è il talento di Ettole Majorana, riconosciuto dai ricoperni come il più geniale, giovani fisico fino all'ora riconosciuto e che ha tutto studiato con lui Ettole Majorana non significò a ipotizzare una particella con delle caratteristiche si fatte ma scrissero si fatte queste poche cose di diaccoli e lo fece da giovanissimo Ettole Majorana produce il massimo della sua produzione scientifica una logica di produzione scientifica è intorno ai 24-25 anni giovanissimo all'epoca trattava con un sistema di otto lato di cerca quindi per entrare nell'accadeno c'era bisogno di un percorso molto lungo come quello di Erno, Frennale, Magistrale, dottorato di cerca, segno di cerca oggi la galetta è veramente molto lunga e complessa ma se i talenti dopo l'area riuscivano ad entrare nell'accadeno nel senso dell'Università italiano prendendo carte d're, avendo carte d're nelle Università italiane giovannissime, i 24-25 anni erano dei tempi molto diversi da poi in attuale quindi c'è una crocchia, è possibile studiare un'altra non so se in alcuni corsi più avanzati si possa denga il proposito o tre le di no per di fatto un corso sulla fisica degli utili che comprenerebbe tutta la parte teorica non è inserita nell'equipum dell'Università grande di partimento di fisica e astronomica e sulla base di quelle di oggi è i successivi teorici del corso dell'ingamnia hanno continuato a sviluppare e a migliorare quindi quello che c'è oggi è di fatto si fondata sull'escritti e sulla prozione di teoria è una stessa e il vostro collega mi chiede ma allora perché? visto che non è stata ancora scoperta il neutrino ovvero le scoperte dei caratteristiche del neutrino come se fosse un neutrino di maggiorarna o gli esperimenti non danno tracce gli esperimenti che mirano a migliorare questi vegetali di 8.0 non mirano a fornire una risposta esastiva con una sfida perché semplicemente da un'idea ancora oggi dopo svariati decenni si continua a seguire questa strada l'idea di maggiorarna non è peribina indipendentemente dei carpoi che possono essere appunto una spegnulazione teorica ma fin quando l'esperimento al metodo di Galileo Galilei non conferma quella ipotesi rimangono delle ipotesi nella interpretazione di maggiorarna del neutrino una caratteristica essenziale, fundamentale e basilate è una massa non nulla del neutrino stesso la dove nella precedente gliela se Benefena hesteggiava, introduzionava una massa molto tuttora tuttavia come detto l'idea iniziale andava nella direzione di una massa nulla che di fatto è stata in qualche modo sancita, sottoscrita da un altro strato delle particelle elementali che attribuisce artificialmente la massa nulla al neutrino la cui conferma più fino ad oggi fino alla resistenza, la conferma della resistenza del Bosonevics è la strada madre e quindi per lungo tempo gli scienziati hanno continuato a credere che il neutrino avesse massa nulla ma nel momento in cui scopriamo il fenomeno dell'occilazione confermato negli anni 2003-2005 da diversi esperimenti in particolare tipo blipo e quindi è del riconoscimento della scoperta con la collaborazione sulle tante cambe in un esperimento lontesline in Giappone ma la scoperta viene in realtà contemporaneamente fatta se Benefena non pubblicata dalla collaborazione macro collaborazione in italiano o Stato d'Unità è apparato l'esperimento al presso il laboratore nazionale del Danzazio per cui cui quel quel gli italiani fanno più conservatori per cui quando trovavano questi dati anomali su l'existenza delle oscillazioni pensarono o sbagliato cioè qualcosa che non va e presero tempo per cercare di confermare effettivamente prima di sceglierla la notizia ma di fatto è che parli l'euro la stiamo parlando dei primi anni in cui cui scoperta delle oscillazioni di Nekino grazie a queste due collaborazioni ufficialmente attribuita allo conservatorio supercaneo grande ma comunque confermata da Narko c'è infatti un articolo che parla della conferma delle disegnazioni del Nekino con tanti didatti e che la conferma attraverso altre tecniche sperimentali e attraverso un altro esperimento quando riuscite a fare il progetto con un esperimento sicuramente differente che utilizza altre metodologiche che riesce a confermarvi quel controllare in stessi insultato sperimentale e la conferma certa della interferonore stesso con le oscillazioni del Nekino non sono possibili se non avvetendo la massa massa non nulla che usino e quindi qualche modo questo consapeva anche di fare il rosario in angelo cioè la teoria di Narko poi che allora oggi sta claro che la massa che Nekino ha di massa e quindi in l'inizio in in sarebbero possibili questo tipo di esperimenti dal momento che il doppio beta zero è permesso per nutrizino di oramica cioè se il nutrizino ha massa ma ovviamente abbiamo di questo vero da cui l'interesse è ancora podierno nel possibile questa strada la misura della ricerca di nutrizino di oramica la transizione può essere rappresentata di questo questo sostanzialmente avete sì una prima misione beta con una misione di nutrizino ma il nutrizino viene riassorbito e viene rassorbito questo che che rappresentato in questo modo e viene rappresentato in questo modo in termini di focule del decadento e da un punto di vista schematico tipicamente si usano questi singoli non l'antigna intera ma l'antigna tratteggiata di indicare una particella che deve essere assorbita quindi non di sìbile non è nuvelabile lo stato finale per tanto beta virtuale questo fa sì che alla fine ciò che bello, ciò che dovrei trovare nella fase finale del decadimento siano due controlli inoltre il processo il processo esotico nel seccio non sono raro e poco e probabilisticamente poco favore tutto ma anche un processo in cui io devo utilizzare i dei forti e assolutamente disportanti rispetto l'epoca in cui fu proposta questa idea non solo ho tenuto un massere di logrino di vessa da zero che oggi sappiamo essere una realtà io ho l'azione del numero meccotronico che ad oggi non è stata mai osservata e spin flip ripartamento dello spin del mio primo stesso inoltre da un punto di vista non entriamo nei conti non è possibile fare non ripeto questo richiede anche conoscenza e matematica molto avanzata ma è importante anche eventualmente avere che caro di riportarlo all'esame avere cheare alcuni punti cosa ventali capire da un punto di vista fisico quelli che sono i legami con le osservabili ovvero con altre esamezze fisiche di globo che possono aiutare nell'esperimento, nella progettazione nell'esperimento a capire comissime su una cosa l'altezza di probabilità per produrre un massacrio virtuale un univiro della prima transizione è proporzionale alla massa questo rapporto la massa del mio tino è da cui massa necessariamente non nulla su un'energia del mio tino stesso l'altezza di probabilità del doppio di carimento delta è quindi proporzionale al qual è altro della massa di tino fino e questo accorta un contributo differente anche in quella che poi la resta inedata negli esperimenti effettivamente sono stati compiuti in quanto questa probabilità può essere misurata e se verifico questa riuscita questa diretta proporzionalità al quadrato della massa del mio tino e oggi sono note le masse di mio tino ora ci sarà una tabella un po' più più vi mostrerò una tabella in cui cui riescono i risultati attraverso i vari esperimenti guardate sempre che stiamo casellando questo argomento nell'ambito delle misure dirette per la massa del mio tino ma poi vedremo anche quelle che sono le delle le le le di di stesso e sono illegate tra lo scempabile e la scoperta e quello che voglio di completamente verificare in un esperimento e sostanzialmente questo questa diretta proporzionalità stella massa del tino è nulla il doppio di carimento delta z delta delta zero come senza emissioni di tino tino non si sterve e non sarebbe possibile distinguere in un corpo solo io con questo tipo di esperimenti verificerei l'insistenza di doppi decadimenti senza inissioni di tino tino accetterei che la massa del tino diventa zero accetterei che è possibile un spin-flip e quindi di conseguenza anche che il mio tino concile con la propria antiparticella quindi che il mio tino è di fatto una particella di margolana il primo esperimento questa è un'altra nota a margine che che avevo già anticipato al nostro olega e che anche in questo caso l'Italia è stata la prima l'assoluto a realizzare esperimenti del genere tra l'altro non ricordavo questo elemento nella strada di stessa e che gli avevo anticipato che il assimo riferimento della fisica italiana sperimentale per la ricerca di doppi decadimenti beta zero è il professor del fiorelline il professor del fiorelline è esercitato fino a poco tempo fa ancora come si troverse la fisica sperimentale all'università di Milano ed è stato il principal investimento di questi esperimenti oggi sfociati e questo mi sto dicendo quindi anche a me li mi mi anche delle informazioni attualissime oggi sfociati nel più grande esperimento al mondo è più raffinato, più performante che è così detto il esperimento cuore dove cuore un acronium sapete che fisici soprattutto in fisica delle particelle e di astroparticelle si divertono a creare degli acronumi che in qualche modo poi ricordino delle parole e dei concetti di altro tipo il esperimento cuore è tuttora in fase di realizzazione un enorme apparato che deve lavorare underground deve lavorare in presenza di silenzio cosmico cioè in situazioni in cui viene tagliato il massimo eliminato, tagliato il massimo delle particelle è un decadimento raro è un decadimento esotico le misure che hanno fatte sotto condizioni supercontrollate della priega Gicosta in prima se eliminando tutto il fiocco dalla landa di particelle che ci investe continuamente dovute al fenomeno degli essenziali e questo tipo di esperimenti che che essere fatti a differenza della ricerca della fisica astroparticellare che abbiamo visto a The Water of Angerai se gli esperimenti capano di internet o a escludo questo tipo di esperimenti devono essere fatti in luoghi sotterranei in galerie sotterranei in caverne molto spesso in ex miniere addonate in Italia abbiamo un accordatore nazionale del transasso che rappresenta un assoluto un sito d'eccellenza per questo tipo di decadimenti in aria l'apparato è un un enorme per quanto un parazzo di 4 piani è stato realizzato un liberatore prototipo, detto coincino che ha già dato per studi di fattibilità un liberatore prototipo fino ad arrivare a quello che il liberatore finale in questo tipo di esperimenti come tutti gli esperimenti in fisica delle particelle e delle astroparticelle occorrono svariati anni di ricerca e sviluppo per arrivare ad utilizzare e che lo si fa anche attraverso la produzione di prototipi il prototipo ha dimostrato la fattibilità dell'esperimento cioè è ben funzionante le sistematiche sono utente sotto controllo ma tu fai questo tipo di esperimenti sono un avvenezzo tra fisici particerari e astrofisici particerari ma con delle competenze aggiuntive perché le metodologie vedremo un esempio nella serare soffrissima le metodologie sono diverse nel senso che questo tipo di esperimenti data la natura e l'evento stesso che vado a cercare devono essere realizzati in criogenia quindi a temperature bassissime quasi vicine al exerzamento e utilizzavano delle sostanze radiative che siamo pubbliche quindi devono controllarlo spinto su quella trasorgente che deve decadere e un controllo di tutti quelli che possono essere le fonti di la utilità ambientale compresi i lagi cosmici prodotti in atmosfera realizzare questi esperimenti luoghi in cui sia silenzio cosmico e avere sviluppare delle tecniche ultra raffinate nel raggiungimento di passissime temperature e passire gli strumenti di dispositivi lavori a passissime temperature che non è banale perché la bassa temperature cale tutta una serie di questo risultato particolare attualmente per esempio con il risultato di Gigi ricerca stiamo facendo rendi su dei fotosensori di ultima generazione a passi e silenzio che sono i fotomotri privatori a silizio dobbiamo farli lavorare a bassa a bassa, ma non criogenia a bassa è il termine di un meno 40 gradi che non è una temperatura estremamente bassa questi rivelatori in particolare andranno nello spazio quindi devono essere soggette di situazioni tecniche non differenti sul satelite o sulla stazione stanziale traduzionale e notiamo che il fotosensore è una vera resiste perché esistono dei risumenti di cellulite climatiche una serie di dispositivi che consentono di mantenere la temperatura controllata pure di azaro a passare la tanto ma quello che non resiste già se sei almeno 20 gradi sono le colle tutto ciò che mantiene l'assemblaggio degli elementi componenti la elitronica tutta una serie di cavi stessi quando fate il cadraggio per poter poi liberare il senale immaginate quando voglio scendere a temperature ben alti sotto e da 50 gradi e bene, ecco le fiorini aperto una strada e aperto una espertisa di una classe di fisici e ricercatori italiani estremamente importante oggi il cuore viene considerato un esperimento per l'impasto di denerizzazione viene considerato un esperimento di infermento del settore per la ricetta del dettaglimento in beta beta zero e per la conferma dell'eutrino di negrodrama e aveva nel 1967 proposto per la prima volta al mondo non più un esperimento contratuale ma un esperimento da concretizzare per liberare il decalimento in beta scendiando a questa sostanza in Germania io provocando il decalimento in selenio e vedersi effettivamente lo stato finale produce il selenio e due elettroni senza traccia di neutrini oppure presenza di neutrini naturalmente siamo agli abbi agli soldi delle tecniche e della concretizzazione dell'esperimento stesso e il primo esperimento non dà la sinistra il segnale assente facendo estrema attenzione rispetto a tutti quelli che erano in un'esittà di questo esperimento il segnale era tuttavia assente guardate bene che quando realizzate un esperimento stia a la ricerca di un evento mai visto dopo il decalimento beta zero a zero ma in realtà se torniamo macro che copriereò a citare perché è strettamente legato con la scoperta delle oscillazioni di neutrino in realtà macro la prima lettera della cosa che postava è monopolo magnetico macro non c'è da realtà per scoprire e definire l'esistenza del monopolo magnetico un'altra grande domanda come sapete non esiste la carita magnetica insolata positiva o vedativa ma alcune teorie in particolare relative alla cosmogilia prevedono esistenza nei primi istanti di vita dell'universo la traduzione di monopoli magnetici ma con dettagli di attività non ha trovato confermato l'esistenza di monopoli magnetici ma anche quando ciò avviene non dovete mai immaginare che l'esperimento abbia fanno il rito assolutamente no il non trovare un segnale comunque il risultato scientifico perché consente di porre un lindesupedone ha flusso atteso il qualcosa di monopoli magnetici nel caso dei primi esperimenti beta-peta-zero di la mia ottima ignorana e così via ci sono tanti esperimenti anche appuramente in corso che non hanno ancora il vado segnale ma poco questi limiti superiori che sono limiti superiori derivati dalla sperimentazione derivati da rissure e che possono essere poi utilizzati quindi c'è un feedback diretto nelle teorie che utilizzano invece gli limiti ipotizzati calcolati e vengono utilizzati nel momento in cui date ai teorici questi limiti fornite teorici, questi limiti sperimentali che servono a raffinare le teorie stesse dunque se ottenere un limiti superiori alla vita metà del Germano e alla massa del Gondino insieme ai tonali, al 1994 vedete, tanti anni passano era veramente un campo inesplorato assolutamente reggine questo della ricerca delle teorie in Oriorano dobbiamo aspettare circa 30 anni perché un altro esperimento ripeta questo tipo di misure nella collaborazione considerata ripeta questo tipo di misure e successivamente la collaborazione Eiderberg-Moscas lo ripeta a sua volta 1998 fornendo dei limiti superiori ancora non otternando l'esistenza del meta per azero ma fornendo dei limiti superiori sempre sul preciusse un esperimento più recente in realtà avrei già anticipato quello che lo stava a quale dell'arte, oggi il giorno l'esperimento cuore fu l'esperimento MIMO-3 condotto alle laboratori soterrane del preciusse nel 2003 studiando la seguente reazione anche in questo questo non si lovarono non fu confermata l'esistenza del beta per azero vero l'assenza nel stato finale ma ci dà forse i migliori risultati su quello che è il tempo di spazzamento di questo tipo di decalmento 30 e 20 anni quindi stavamo parlando veramente di numeri e di ordini di grandezza un cuore dalle ordinelle rispetto ai limiti e ai quali sono abituati e un valore al netto per la massa del neutrino senza specificare il sapore 30 e 20 30 e 20 un esempio non è per ordine il netto, non è possibile farlo per tutti gli esperimenti ma per farvi capire la complessità della realizzazione di un esperimento del genere e in ordini dimensioni esperimenti andervi complessità articolazione degli apparati notevole di numerosi anni non solo ricerche si può preparare anche i realizzazioni dei blocchi di spiccia dell'apparato stesso e tipicamente poi questo tipo di esperimenti hanno come obiettivi con laterali altre ricerche perché di fatto possono andare a rivelare volendo rivelare questo sistema di pratico e di questo caso possono fare anche altro basta ispensare gli esperimenti una fisica del neutrino oggi in cossò un generale se si è una fisica del neutrino, un pratico e un alco che del neutrino è un assoluto pratico e un alco cioè se volessi avevi altri filominici che non sono quello principale allora a questo punto della storia e con questi elementi in realtà possiamo conciare a costruire quella che è una tabella relativamente inutile alla massa del neutrino da misure dirette dove le misure dirette, mi ricordo un attimo per fare un recrato di quello che ci siamo detti l'altro giorno, le misure dirette sono di tre tipologie, studi cinematici in laboratorio In realtà, in realtà, in realtà, in realtà, in realtà, in realtà, in realtà, in realtà, in realtà, in realtà, in realtà, in realtà, in realtà, in realtà, in realtà, in realtà, in realtà, in realtà, in realtà, in realtà, in realtà, in realtà, in realtà, in realtà, in realtà, in realtà, in realtà, in realtà, in realtà, in realtà, in realtà, in realtà, in realtà, in realtà, in realtà, in realtà, in realtà, in realtà, in realtà, in realtà, in segnale ha permesso una serie di studiati conferme di vari interdisciplinari, dalla comprensione dell'evoluzione di una supernova, alla fisica del neutrino, alle sorgenti astrofisiche, perché le supernova erano state utilizzate dai ricofermi come possibili sorgenti di laggi cosmici di energia elevata nella nostra galassia, ha costituito davvero un momento molto felice per la fisica tutta. E tornando a questa tabellina, vedete qui ancora non è completa con i dati attuali relativamente al neutrino da Oolimers, che sono state l'ultimo ad essere rilenato. E il... ah, però qua, ok, comunque, c'è... si sono riportati i valori del momento magnetico, e nella sezione tutto sul nucleo è un'etica più importante per fare questo tipo di studi. Qui, non compare, con questo un altro, non è vero, è che questa versione è stata rinquinata in qualche modo, soprattutto quella che è la proiezione attraverso questo sistema, cui chiaramente, ma non riportati, quindi riporterò i numeri, completerò questa tabella nel successivo mio, completando, li possiamo ripostruire fondamentalmente da quello che abbiamo avuto, vi darò dei numeri più attuali, su quello che la massa superiore, la massa del neutrino elettronico è muonico, gli ordini di grandezza, anche nei limiti superiori più gravi e recenti, rimangono ordini di grandezza, di differenza tra il neutrino elettronico, il neutrino muonico è elettronico, è elettronico talo. Dunque, allora, un modo, non lo speriamo, allora facciamo così, anche l'ordine, è meglio attualizzarlo, ovvero, nella ricerca della massa del neutrino, dei neutrini, le misure indilette, ovvero esperimenti che cercano altro, nello specifico la conferma delle oscillazioni del neutrino, consentono di fornire una valora superiore, in particolare, lo studio delle oscillazioni di neutrino nella materia, però questo recessi da per capire questa tipologia di misure indilette, necessita chiaramente di elementi sulle oscillazioni del neutrino, che vi fornirò, ma perciò di questo punto, fare una un'impestione di argomenti, lasciare una, come se non stessero i quali sono l'oscillazione della materia, e per capire l'oscillazione della materia e per capire come verificare l'esistenza delle oscillazioni della materia, è vero che apriamo una parentesi un po' più ampia, questo ci consentirà anche di riprendere il punto tre del nemico sul neutrino, che è la misura di un neutrino di estansione del supernore. Questa parentesi molto più importante, mi deve, è utile per farvi capire un'altra, questa non è fondamentale, che è la seguente, sempre da, ed è legata anche alla domanda, ancora all'interio le domande che il vostro collegamento posto, alla fine della riusione precedente, il campo, signorier, mi prego di fare, non porto il vostro nome, però mi prego di fare silenzio, se è necessario non cararvi, fatelo, non c'è problema, non mi offre, però parlare crea un uomo di foglio antipatico, non sono fermi ma credo anche a credo di sicure, tuttavia, che guardate che siete tipe, l'università è tipe, dovete entrare e uscire, non ha una università, e qua mi ha provato niente, senza distrubare, io sempre mi studente, magari non do la pausa, ma se siete stanchi, non cararvi a nativa prendere un caffè, telefonate a fidanzare a fare quello che volete, ma non dispubate la riuscita, ok? E non sarò il dì che lo potete uscire, questo vanno a riuscire, d'accordo, siete limpe e riuscire quando volete per le vostre esigenze. Allora, torniamo noi, un aspetto fondamentale è legato, o perché è la matura riuscita del neotino che questo mai lo stiamo capendo, ma anche ha fatto che il neotino può essere rivelato, insurato, studiato attraverso canali differenti, li chiamiamo, in giro, canali, che cosa si vuol dire? Neotini vengono prodotti, intanto, a livello di produzione, come vengono prodotti, da cosa vengono prodotti, possiamo osservarli sorgenti di neotino in sorgenti naturali e sorgenti artificiali, e questa è la prima grande classificazione che si fa. I neotini possono essere sorgenti naturali, possono essere prodotti in maniera naturale, in fenomeni naturali, in natura, la natura ci fornisce neotrini di tutti i tempi, e qui le le chiano del punti, sorgenti naturali sono la teta, il nostro pianeta è sorgente di neotrini, questi neotrini vengono detti geonetrini, ovvero neotrini prodotti dal pianeta terrestre, l'interesse per i geonetrini è recentissimo, si ipotizzava l'esistenza di neotrini rivelabili prodotti da corna, l'ubrio della terra, ma di fatto pensate che questa è proprio una ricerca recentissima, il primo esperimento da aver fornito la prima nissuta di neotrini prodotti dal corte, il restre, dal nucleo del pianeta terra, l'esperimento Borixino. L'esperimento Borixino è un altro grande, importantissimo esperimento realizzato negli anni 2000, a chiuso, a chiuso un 5 anni fa circa, un enorme sfera grande quanto un palazzo di due piani, ma è imaginata questa sfera giardalvesca, collocata ai laboratori nazionali del Grazazzo, piena di cintillatore di liquido, è tappezzata, interamente ci sono delle foto nel avvioso, Borixino ottieni un po' la propria di mezzo, a un certo punto, quando ha confermato alcuni dettaglenti relativi ai cicli di fiacenio del sole, Borixino nasce per rivelare neotrini solari e per diminuire definitivamente la questione del flusso atteso e flusso misurato di neotrini proletari del grazone, ma riesce come la gran parte dei rivelatori fisica del neotrino a rivelare anche altre foto di neotrino, e in particolare Borixino pubblicato per la prima volta, un tre anni fa, tre o quattro anni fa, la nissuta di neotrini provenienti da corte a peste, il pianeta terra è costituito da un nucleo febbroso a temperature e pressioni elevatissime e sostanze raveltive, udario, rutorio, torio, che decadono e nel loro decadimento sono decadimenti beta, producono neotrini, il neotrino è stato dotato di una sezione molto bassa, come abbiamo visto, in grado di attraversare tutta la terra da corte alla mantello fino alla costa terrestre e poter essere libenato in esperimenti del genere, perché è grande il grano al solito, perché devo essere certa che la fonte è il corte terrestre e non nell'atmosfera o soggetto, è granatica o extra granatica. Quando io detto che il neotrino rappresenta una sonda potentissima, lo è a tutti gli effetti, non solo quando ci porta informazioni, quindi è uno dei cosiddetti, in maniera fascinante, uno dei cosiddetti messaggeri del Cosmo, perché ci porta informazioni dalla stella più vicina, ovvero il sole, dall'esplosione delle supernova, da soggetti extragalattiche, particolarmente violente, come nuclei grafici attivi, eccetera, ma ci porta anche informazioni dalla corna del piadenta terra. Pensate che tutto quello che fanno geofisici, che cercare di strutturare, studiare, misurare la struttura della terra stessa, è infatti come tecniche, principalmente nell'accezione più grezza, malalmente, dei carotaggi nel terreno, ma chiaramente i metodi di indagine geofisici non consentono, non consentiranno mai di arrivare ad avere informazioni su quelle che sono realmente i stati interni del piadenta terra. Misurare un flusso di genutivi ci dà delle informazioni molto importanti, su da struttura della terra stessa, che sembra oggi una cosa nota, la studiata fin dall'escola elementale di un immagino nella oligografia astronomica, ma che è realtà, tante parti anche delle strutture geofisiche non sono ci confermate, ancora confermate, da reggisime, ma nello stesso tempo queste emissioni di imprime è legata anche a quella che è un flusso tecnico, a quella che è un'emissione tecnica degli stati più interni, la cui conoscenza e la cui misura sarebbe molto importante riuscire a capire qual è l'emozione del piadenta terra, da cui ai prossimi segoli. Dopo Borexino si è aperto un filone di cerca sulla vistula dei genutini, ed oggi l'esperimento attuale che può fare questo è fare in maniera molto attugata, è invece un altro grande liberatore, un altro grande conservatorio che è un conservatorio Juno JUNNO che è un anche un enorme spero di scintillatore liquido, da pensato di posto di applicatori nella sua proprio schematizzazione più cruda, posizionato all'interno di una carenna e in una regione particolare della Cina circondato da una conservatore fisica del genutino prodotto da rettori nucleari circostanti come gond principale, ma una serie di gond scientifici secondari, la misurazione di genutini appostensi, la misurazione di genutini da esposione di subendoma, è la misura di genutini di genutini, il flutus pulonico, questo senso, sarà il genuno, l'installazione l'RND è durato circa una decina da anni, qui in un dipartimento di fisica spromia abbiamo un gruppo di ricerca che lavora in giro, tesi di laure e tesi di dottorato già prodotte e proposte e insottomesse per questo tipo di lavoro, Juno, alcuni mesi fa è stato ottimato il riempimento, non solo l'installazione meccanica fisica della sfera stessa, ma il riempimento di scintillatore liquido della sfera stessa, e iniziato con una cresa nata, quindi quindi ci aspettiamo da qui ai prossimi anni un fiotto di dati di tutti i tipi che possono aiutare in assoluto, sia la fisica del neutrino da oscillazione di da reattore che altri anni di fisica di meccanica stessa, Neutrimi appostensi, i neutri appostensi sono invece poi però un neutrino appostrero, se non si avete molte cose le sto dicendo, poi scorriamo le slide seguenti, poi l'idea ho strutturato i neutri immaginando proprio geograficamente, da core della terra, sole a soggetti naturali molto più distanti, i cosiddenti neutri appostensi, non fate confluzione, sono i neutri prodotti nelle interazioni in atmosfera terrestre durante lo sviluppo delle stessi del Shadows, che era presentare le prime lezioni, che con le slide sono state rilasciate tutti, vi ricorderete che protoni particelle calde da protoni a ferro che giungono sorgenti galattiche e extragalattiche, se particolarmente energetiche, in particolare con un energie superiori a 10 a 17 e a 10 a la metode di trombone, impattano le strade arti della prospera, 3,30 e 50 chilometri, producono che la prima interazione di tipo stochastico è una tritura a spallezion, spaccano le molecole di ossigeno e azote in atmosfera, producono a baranga questo ciam e questa cascada, detta anche di praticare le secondane. Le prime, la spallazione di un creale, lo capiremo meglio quando faremo la fisica erronica, ovvero capiremo come le prodotti caoni più in cega sono caoni, pioni, pioni tra i 0 decali da una gamma che da l'uovo alla prima presenza di fotoni e quindi alla cascata elettromagnetica, i pioni decalano per più che meno i muoni appena compaio i muoni, compaio i neutri di muoni. Interazioni di corponero di elettronici avete neutri di elettronici, quindi avete una grande quantità di neutri di elettronici e muonici prodotti in questo fenomeno delle istenze di Schaus. Nentre tutte le altre particelle, poi, beh, poi bisogna le mescugare nel specifico come si sviluppa effettivamente questa enorme cascada, questo grande ciam che non può essere prodotto in laboratorio, in nessun laboratorio del mondo, le cui la servono in genera. Il dal punto di vista nucleare dal punto di vista particellare ha un certo punto del fenomeno si arresta quando l'energia del singolo particella decresce al di sotto della soglia necessaria a mescare le successive interazioni. In qualche modo il numero di particelle nello sciame di nuovice, senza male, ha già assi completamente, ma c'è una componente detta penetrante che è quella di particelle che riescono invece ad arrivare fino a terra e addirittura andare sotto terra, in primis muoni, quando spesso si parla di fisica dei langicosmi, se si parla di misure a livello del mare o a livello a quote base, in realtà, si avvolgano di fisica dei muoni e aprono poi un altro capito ma che non sono i langicosmi disciplinari prodotti dal resto del universo. Ma le particelle che sono interessanti continuano ad essere nuove. Mentre i muoni si arrestano entro tre... trecento metri, dipende dall'energia ma fondamentalmente tanto vero perché il livellare è abbastanza enterrare, una serie di piano disciplinatore e poter quindi tracciarvi, bloccarli e livellarli e ne utrili atmosferici attraversano la terra e possono riaffiore dalla parte opposta, da cui una serie di esperimenti che vengono disuzionati nel modo per livellare neutriloivitivo atmosferico. Questi neutrili non ci danno informazioni sulle origini e le soggetti del universo ma ci consentono di studiare molto bene quelle che sono le interazioni di l'ultima stessa, quindi sezione di tutto, lunghezza di interazione, ci consentono di studiare anche le oscillazioni perché un modo per misurare le oscillazioni è per esempio avere un esperimento come altro, come il supercannio grande che misura il flusso di neutrini, immaginando che questi neutrini stiano arrivando dall'alto dello sciame, e misura il flusso di neutrini che arrivano dalla parte opposta della terra, del globo terrestre. Maginate una situazione del genere in cui avrete un apparato disuzionato underground su un livello alcun sulla superficie terrestre, il sciame provoca il grande costo di coprimandio, che è un'altra è questo sciame, e questo sciame vede una serie di neutrini, è possibile misurare un flusso di neutrini che arrivano dall'alto verso il basso, ma questo accade anche dalla parte opposta della terra, quindi ci saranno un fascio di neutrini inisturbati e che riesce ad alplavizzare l'atmosfera terrestre, i cosiddenti down-going che vanno dal basso verso l'alto e gli up-golder che arrivano dall'alto verso il basso, e questi flussi diversi, se il flusso superior e diverso dal flusso inferiore, state certificando resistenza alle accelerazioni del neutrino, se quei misurati di neutrini elettronici, e il flusso di neutrini elettronici dal basso, è diverso da questo, significa sostanzialmente che neutrino, trascendolo dalla massa terrestre, un materiale molto denso, si è trasformato in altri tipi di neutrini, e quindi voi avete un flusso inferiore, avete un deficit di neutrini, di neutrini che è una certificazione delle oscillazioni stesse, quindi le otrinazioni di un specie vengono utilizzate in campo sperimentale la riligrazione di un otrinazione di un specie particolare per studiare le caratteristiche, le proprietà del neutrino stesso e le oscillazioni. Neutrini solari prodotti da soli, il sole si mantiene in vita grazie a le azioni termine mondiali globali del ciclo tipi e del ciclo cnl, nel quale adesso vedremo, tutto questo sarà poi sviscerato nelle successive slals, vengono prodotti nei neutrini. La riligrazione dei neutrini solari ha costituito il primo mezzo della comunità scientifica per studiare le caratteristiche del sole, per studiare in particolare le reazioni nucleariche avendono all'interno del sole, e ha avvetto una domanda, anzi, una diatriba, inizialmente una diatriba fondamentale tra i piscici solari e i piscici del neutrino, poiché il flusso atteso di neutrini prodotti nelle catene ccpp e cno è governato dal cosiddetto Modello standard del sole di Bavolla, che dava un determinato numero, e quando sono state effettuati una serie di esperimenti a terra, a stima terra, per rivelare questi neutrini solari, sistematicamente si trovava il flusso inferiore. Ovviamente la comunità di fisica solare credeva a tuttura quello che era il Modello standard di Bavolla, i fisici invece erano anche ad Alcè, che era la fase sonorata negli anni 90, il primo esperimento, questo senso fu home state in un miniero abbandonato all'estate uniti d'America, ma il grande esperimento è stato realizzato in Europa, in esperimento da Alex, ancora una volta un apparato presso il laboratorio nazionale del transasso. Il flusso misurato era sistematicamente diverso, inferiore al flusso atteso, si vava nel problema del deficit di neutrini mancanti, e corte il flusso è capito che erano mancanti, perché i neutrini, tonodocchi nelle capelle più piecene, nel coro del sole, attraversano, di una certa natura sono neutrini elettronici, nell'attraversare il mezzo, estremamente denso della nostra stella, e affiorando alla superficie, poi attraversando la distanza, 150 milioni di chilometri, medica la terra e il sole, si trasformano, converto in altri tipi di neutrino, per cui a terra io non velo effettivamente il flusso originario e nezzo. Avevamo ragione bravi, nel senso che il modello di Bacol, il misurale, che mi calcolava un flusso effettivo di neutrini elettronici messi dal sole, ma nel frattempo c'è fenomeno delle oscillazioni, oscillazioni nella materia che tra l'altro sono amplificate, cioè il fenomeno del tono di dissonanza, nella materia queste oscillazioni possono essere più intense, ovvero più rapidamente il neutrino si trasforma in una specie opposta, ma c'è anche oscillazione in materia era refatta, in spazio che non è vuoto, ma non è neccun'è estremamente denso, di separazione tra il sole e la terra. A terra mi solavamo il flusso in fedore, dopo qual fatto che nel frattempo era avvenuto le oscillazioni, e quindi parti, le neutrini elettronici, si erano trasformati in altri tipi di neutrino. Con la scopri, la sonda di neutrino è l'unica che ci consente di avere delle informazioni molto accurate sulla natura del sole, in particolare, tutti gli studi riguardo la colma solare, solare, plasma, eccetera, che sono appannaggiate nella fisica solare, possono essere effettuati con una serie di sonde. Quando vuole andare a investigare le reazioni termine di globali, e questo fa parte della branca detta stronfisica nucleare, devo necessariamente fare esperimenti che mi rivelino i miei piste. E esperimenti, l'ultimo grande esperimento, appunto è l'esperimento Borsino, che ha dato le emzioni più raffinate della produzione di neutrino, non solo del ciclo più primante del ciclo ceneo, in particolare di assurda, rami secondari, e ci aspettiamo delle nuove informazioni anche dall'esperimento giusto. Neutrino è da esposione di supernova, da supernova di supernova, quindi voi avete un'inferinatura di astrofisica, immagino, perché avete elementi di astrofisica a terzo anno, probabilmente l'avrete già seguito e fatto. Non so qui a recuperare queste informazioni, nel senso, è lo stato finale di STM, in condizioni particolari, cioè, è un punto di vista, non astrofisico, ma da punto di vista di fisica particellare, e astrofisica particellare, i supernova sono sorgenti molto interessanti, molto importanti, perché riescono uno ad accelerare, raggi cosmici, ovvero protoni e nuclei, fino al febbro, a energie molto elevate, così come avrebbe utilizzato in picoferniche, che abbiamo atturato e accettato con l'esperimento di Erni, ma comunque anche un fiotto di neotrini che hanno delle particolari caratteristiche. Nisurare i neoclibi provenienti dalle supernovae ci consente, uno, una misura diretta della massa di unulino, due, di cattire a cui i meccanismi non chiarino, non ancora chiarino l'appoggiungista astrofisico delle evoluzioni della stella, stessa delle esposioni di supernova, e tre, ci consentono di avere una sorgente di neoclibi naturali, che è sufficientemente conosciuta, misurata sulle supernovae, c'è una comunità intera di astronomia astrofistica che prosegue con i polisculi. E le eotrini da supernova vengono, il più grande esperimento di riferimento è stato l'esperimento e le vudi da un detetto, anche esso considerato nel lavoratori nazionali del Gran Sasso, che ha operato per circa 20 anni, purtroppo non era ancora attivo nel 1987, e anzi, Nathque, in seguito al grande successo delle misille eseguiti alla esposione della supernova, in 1987, e non ha acquisito perché il riferimento di esposioni delle supernovae circondale della nostra canazia è molto bassa, una circa ogni 30 anni, in realtà ci aspettiamo che possa averne qualcosa nel prossimo che vi do, e l'esperimento giusto fa anche questo, riferensione di neoclibi in esposione di supernova, esistono, qui la ta manca, ma ancora la linea che vi aggiungerò, esistono i neotrini di origine nel termologio, c'è una classificazione tra neotrini cosmogenici e neotrini astroparticennali, c'è una linea che la vi aggiungerò, nella ricorda di voce, ci sono neotrini che vengono prodotti nelle interazioni, nello spazio intergalatico ed extragalatico dei raggi cosmici, di energia molto elevata, quindi sono frutto dei raggi cosmici primari, non nell'accostera terrestre, ma nella loro propagazione dell'unidesto dalla soggente astrofisica a noi, sono considerati ultra high energy neotrinos, energini di altissima energica, che sono quelli che ha misurato a S-Tube, quelli che sta misurando del ordine del PEV, KM-TREMET, e la cui produzione è estimata, calcolata, ipubizzata e ci aiuterà, scusate, ci aiuterà a capire meglio anche della fisica del raggi cosmicicco in gestre. Sto parlando un po', lo sopra, lo sopra, lo sopra, lo sopra, lo sopra, i neotrini, qui risultati invece, quindi troverete una liga in più con i neotrini di cui io ho parlato, esiste un altro soggente di neotrini, che di il ben stesso, si potizza l'esistenza di un foglio di neotrini, di origine primolitiale, prodotti nelle fasi successive a un primo stato di l'italiano universo, io avevo registrato un'agenia epologica nella mia slide di PEV, in tante alle prendere quella fondamentalmente, a un certo punto c'era proprio un'indicazione trasparenza del neodino, cioè il momento in cui le neotrini nascono come materia. Questo fondo di neotrini còfnici dà un tutto vista concettuale molto simile a cosmico e degram, il fondo còfnico di fotoni FREDI da 2.7 K, che perbesso è un ufficioso, che è stato verificato, la consistenza è stata verificata da Pensiesse Urizo nel 1964, mentre per quanto riguarda le neotrini primordiali, ancora Ogini si sta cercando, ma non abbiamo conferma. Per nell'armi, io con le tecnici nutri affinate, con certamente esperimenti profondamente diversi, dà poi i fatti fidona, l'interesse della scelta delle neotrini primordiali è recentissima. Allora, allora, la, i neotrini invece, di tipo artificiale, la torta sessuale, ci sono neotrini ancora poveri, terrestri. E quindi, possono essere prodotti da reattori neotrini, un reattore nucleare nel suo funzionamento e utilizzando soggetti radiative in un progetto neotrino che avranno meglio il disedito, e i reattori neotrini ben utilizzati in un appunto vista della fisica delle particelle, della fisica del neotrino, non è già capriciale. Tidicamente in questi casi, in livellature, conoscono un specchio di emissione energetico delle neotrini da reattore, è opportuno monitorarlo, quindi questi esperimenti consistono in un livellature principale che posta una certa distanza dal colore del reattore nucleare, una distanza di pare 30, le 10-20 metri fino ai 100 metri, massimo 1 km, e vengono detti, per tanto, lo vedete nelle prossime sline, gli esperimenti shot baseline su corta pazze, ovvero, per una distanza non troppo elevata dalla sorgenda estessa. È un livellatore secondario di quel reto che è realmente appuntamente per il reto l'estesso, perché sembra monitorare il flusso di neotrini in prodotto. Devo essere molto certo cosa sta producendo il mio reto per poter effettivamente ridere se alla distanza dove viene il posto il livellatore avvalle, si verifica il fenomeno di oscillazione. Fondamentalmente, non c'è nessun flusso, ed è quella del sorario flusso diverso avvalle rispetto al flusso a multa, che mi detengono ancora una volta, mi dà una conferma delle oscillazioni del controllo stesso. È la possibilità di risurare dei parametri che sono tipici del fenomeno dei neotrini, e che mi permette di portare delle informazioni sul fenomeno, sulla massa del controllo stesso e superare il miscoramento. Analogamente si può far con gli acceleratori in quanto posso fare un livellatore a grande distanza dall'acceleratore da un certo anno di chilometri a mille chilometri, ancora una volta per misurare un flusso avvalle di destra per il flusso del controllo dagli acceleratori e quindi identificare il fenomeno di oscillazioni stesse. Queste diverse fonti di neotrino le vediamo sviluppate nelle successive slide. Allora, ragazzi, devo fare un pausa che non riesco a parlare perché non mi divergo a cosa. Quindi approfittiamo tanto il flusso di destra per il nato, ora continueremo nelle strade successive per ogni tipologia di controllo in realtà per le informazioni di più. Facciamo per favore 5 massime, 10 minuti di pausa per voi, ma soprattutto per me perché altrimenti non riesco a aggiungere una parte per te. E ci vediamo tra una regina di ultimassime