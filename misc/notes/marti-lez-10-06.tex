\documentclass[11pt, oneside]{article}   	% use "amsart" instead of "article" for AMSLaTeX format
\usepackage{geometry}                		% See geometry.pdf to learn the layout options. There are lots.
\geometry{letterpaper}                   		% ... or a4paper or a5paper or ... 
%\geometry{landscape}                		% Activate for rotated page geometry
%\usepackage[parfill]{parskip}    		% Activate to begin paragraphs with an empty line rather than an indent
\usepackage{graphicx}				% Use pdf, png, jpg, or eps§ with pdflatex; use eps in DVI mode
								% TeX will automatically convert eps --> pdf in pdflatex		
\usepackage{amssymb}
\usepackage{amsmath}

%SetFonts

%SetFonts


\title{Brief Article}
\author{The Author}
%\date{}							% Activate to display a given date or no date

\begin{document}



\section{Tecniche di rivelazione}
\subsection{I geoneutrini}
Il primo esperimento che ha misurato i neutrini terrestri è l'esperimento KamLAND. Questo esperimento nasce come esperimento per la rivelazione di neutrini atmosferici, ma come spesso accade se un esperimento è progettato in  maniera "at large" può rivelare proprietà di neutrini di altra natura (ciò non è per nulla scontato in quanto ognuno di questi tipi di neutrino caratterizza un intervallo specifico in termini di energia).\\
I neutrini di origine terrestre provengono direttamente dal nucleo e nascono dal decadimento $\beta$ di elementi come Uranio, Torio e Kripton. Questi neutrini attraversano, con bassa sezione d'urto, materia molto densa e possono essere rivelati sulla superficie terrestre. Misurare il flusso di neutrini terrestri ci da informazioni sull'energia termica prodotta all'interno della Terra e costituisce la sonda più efficace per studiarne la struttura interna.\\


\subsection{I neutrini atmosferici}
La questione dei neutrini atmosferici è molto interessante perché ha consentito la conferma delle oscillazioni. I neutrini di questo tipo sono prodotti nel fenomeno degli exstensive airshowers: sono prodotti quando i raggi cosmici impattano con gli strati più alti della nostra atmosfera (tra i 30 e i 50 km) e inducono reazioni che coinvolgono i decadimenti di pioni e kaoni in elettroni e muoni con i corrispondenti neutrini o antineutrini. %SLIDE 46
Nello sviluppo degli airshowers l'energia diminuisce mentre aumenta il numero di particelle, a un certo punto si raggiungerà un'energia di soglia oltre la quale non si innescano i decadimenti successivi (In particolare al livello del mare arrivano solo i muoni). I neutrini, contrariamente alle particelle cariche, arrivano a terra senza interagire e possono poi, a seconda della loro energia, attraversare la Terra. Pertanto esperimenti dedicati alla rivelazione dei neutrini atmosferici possono rivelare sia neutrini provenienti dall'alto che quelli provenienti dal basso (ovvero quelli che hanno attraversato la Terra).\\ 
Quando le energie delle particelle degli sciami secondari sono inferiori a 2 GeV quasi tutti i pioni e i leptoni saranno decaduti e restano solo i muoni più penetranti e i neutrini. Gli esperimenti per la rivelazione di neutrini atmosferici mirano a misurare il rapporto tra i neutrini muonici e quelli elettronici: 
\begin{equation} 
   \frac{\nu_\mu + \bar{\nu}_\mu}{\nu_e + \bar{\nu}_e} \approx 2
\end{equation}

Le variazioni del flusso misurato rispetto a quello atteso sono dovute al fenomeno dell'oscillazione del neutrino. Gli esperimenti sui neutrini atmosferici hanno come scopo la verifica delle oscillazioni del neutrino e lo studio delle interazioni. \\

Gli esperimenti Kamiokande e Super Kamiokande utilizzano un rivelatore sito nella miniera di Kamioka (Giappone) costituito da 11200 fotomoltiplicatori che catturano la luce Cerenkov emessa dalle particelle dei raggi cosmici secondari al minimo di ionizzazione in un mezzo scintillante. In questo tipo di esperimento è possibile riconoscere la "firma" del neutrino stesso e misurare il rapporto tra la componente elettronica e quella muonica. Quello che osserviamo è un deficit di neutrini dovuto alle oscillazioni: 

\begin{equation} 
   \frac{(\nu_\mu / \nu_e)_{sp}}{(\nu_\mu / \nu_e)_{th}} = 0.61\pm 0.06.
\end{equation}

Possiamo allora calcolare due parametri caratteristici: la differenza dei quadrati degli autostati delle masse del neutrino e l'angolo di matrice di mescolamento.

\begin{equation} 
	 \Delta m = 5.22 \times 10^3 \,\text{eV}
         \theta \approx 90^\circ
\end{equation}

L'esperimento MACRO (Monopole, Astrophysics and Cosmic Ray Observatory) è stato il primo apparato posizionato all'interno dei Laboratori Nazionali del Gran Sasso e ha confermato l'esistenza delle oscillazioni. Qui riveliamo sia neutrini provenienti dal basso che  neutrini provenienti dall'alto, i quali  devono attraversare 2000m di roccia, per poi interagire all'interno dei diversi rivelatori dell'apparato.\\

\subsection{I neutrini solari}
L'esperimento Borexino ai Laboratori Nazionali del Gran Sasso è dedicato alla rivelazione di neutrini solari, è un'enorme sfera di scintillatore liquido tappezzata interamente da fotomoltiplicatori. Questo esperimento ha permesso la rivelazione dei neutrini prodotti nelle catene PP e CNO, ma è stato anche in grado di misurare con maggiore precisione i geoneutrini. \\
Le oscillazioni del neutrino furono ipotizzate dall'italiano Bruno Pontecorvo nel 1968, ma dal punto di vista sperimentale nascono dalla ricerca sui neutrini solari. Una volta noto il modello solare standard (SSM) di Bahcal e quindi l'esistenza delle catene nucleari PP e CNO (durante le quali avviene l'emissione di neutrini), i fisici hanno deciso di misurare i neutrini provenienti dal Sole. \\
%SLIDE 51
%%SLIDE 52
I neutrini vengono prodotti in diverse reazioni e lo spettro energetico per le diverse reazioni è calcolato attraverso lo SSM e copre da 0.1 MeV a 10 MeV. Il tempo impiegato dai neutrini a raggiungere la Terra è 500 s (molto minore rispetto a quello impiegato dai fotoni che è circa 1 milione di anni). Questi neutrini possono essere rivelati in esperimenti con target massivo e possono essere separati da quelli atmosferici andando a fare misure underground. Si misura un flusso di neutrini inferiore di quello previsto dallo SSM, questo deficit è spiegato dalle oscillazioni.\\ 
Le tecniche utilizzate nella rivelazione di neutrini solari si possono dividere in due classi complemnentari:

\begin{itemize}
	\item esperimenti radiochimici: utilizzano sostanze di riempimento dei rivelatori con particolari caratteristiche chimico-fisiche che inducono decadimenti beta o altre interazioni;
	\item esperimenti elettronici: sono grandi apparati (come Kamiokande) dove si usa uno scintillatore liquido, si provoca uno scattering elastico tra il neutrino e il mezzo e si rivela la luce di scintillazione o Cherenkov;
\end{itemize}

%%SLIDE 55 confronto tra i due esperimenti

\noindent Il numero di eventi osservato è: $R = \Phi \cdot \sigma \cdot \eta \cdot N_{\text{nucleoni}} \approx 10^{10} \cdot 10^{-45} \cdot \eta \cdot N_{\text{nucleoni}}$ (con $\Phi$= flusso, $\sigma$= sez. d'urto, $\eta$= efficienza di rivelazione). 

%%SLIDE 57 qui e a seguire ci sono degli esperimenti ma non li ha discussi nel dettaglio

\subsection{I neutrini da esplosione di supernova}
L'esperimento LVD (Large Volume Detector) ai Laboratori Nazionali del Gran Sasso è un rivelatore esclusivamente dedicato alla rivelazione di neutrini da esplosione di supernova.  È un esperiemnto di tipo elettronico costituito da un insieme di contatori a scintillazione. 

\subsection{I neutrini ultra high energy}
NOTA: spesso sono chiamati impropriamente neutrini cosmici (spoiler anche lei li ha chiamati così nella lezione di prima) perché poi bisogna fare la distinzione tra cosmologici e cosmogenici. \\Sono neutrini provenienti dal resto del cosmo e vanno distinti da quelli prodotti nel Big Bang. 
Per studiarli servono rivelatori appositi detti "telescopi" che sfruttano enormi quantità di acqua, terra o ghiaccio come target massivo. Pur non essendo telescopi classici possono darci informazioni sulle sorgenti. \\
L'esperimento Ice Cube in Antartide usa stringhe di fotomoltiplicatori nel ghiaccio, l'idea è che questi neutrini (che hanno rate atteso molto basso) interagendo con il ghiaccio producono collisioni al seguito delle quali i secondari emessi produrranno luce di scintillazione che viene rivelata.\\
L'esperimento KM3Net ha immerso stringhe di fotomoltiplicatori nel Mar Mediterraneo, il network finale occuperà $3\, km^3$. Questo esperimento ha recentemente rivelato un neutrino con energia dell'ordine del PeV. È complementare ad Ice Cube in quanto ciascuno può dare informazioni solo su uno degli emisferi. \\
I telescopi mirano ad identificare i neutrini prodotti dall'interazione dei raggi cosmici con il mezzo interstellare ma anche a identidicare i neutrini prodotti da sorgenti, tuttavia, per identificare le sorgenti servono più esperimenti distribuiti sul globo. Abbiamo altri due progetti di questo tipo nel Mediterraneo: ANTARES (a largo della Francia) e NESTOR (in Grecia). KM3Net sarà il più grande telescopio sottomarino ma sfrutterà anche il segnale di questi apparati secondari per controllo. \\
La proposta futura è fare rivelazione di neutrini ultra high energy dallo spazio. L'idea è la seguente: mettere in orbita dei telescopi Cherenkov o di fluorescenza che guardano dall'alto verso il basso gli sciami iniziati da neutrini (anziché da raggi cosmici) che sfiorano la crosta terrestre. Questo permetterebbe di aumentare considerevolmente la statistica perché il campo di vista è molto più ampio. 

%%SLIDE  71 SULLE SORGENTI E LE ENERGIE Rossy dice di memorizzare gli ordini di grandezza 


\subsection{I neutrini da sorgenti artificiali}
L’importanza di fare esperimenti con neutrini prodotti negli acceleratori o nei reattori nucleari sta nella possibilità di controllare la sorgente: sappiamo la direzione di arrivo, la quantità di neutrini prodotti e il loro flavour (per i neutrini di origine naturale possiamo solo fare stime basate su simulazioni). Questi esperimenti esaminano intervalli di energia complementari rispetto a quelli visti finora e permettono di ottenere con precisione i parametri che descrivono il fenomeno dell'oscillazione. Per i neutrini da reattore o da acceleratore possiamo decidere gli intervalli energetici da esplorare, basandoci sul principio di indeterminazione: 

\begin{equation}
	\Delta x \geq \frac{1}{\Delta p} \approx \frac{2E}{\Delta m^2}
\end{equation}

\noindent $\Delta x$ è detta "base" (gli esperimenti poi si differenziano in "long base-line" e "short base-line") ed è la distanza tra la sorgente e il rivelatore. L'intervallo spaziale è legato al rapporto tra l'energia e il $\Delta m^2$ che è la differenza tra i quadrati degli autostati di massa. Quindi fissando la base vado a determinare quale ordine di grandezza di $\Delta m^2$ posso andare a studiare. Vediamo un po' di nomenclatura.

\begin{itemize}
	\item Esperimenti di apparizione (o comparsa): usando un fascio contenente prevalentemente un sapore di neutrino si cercano sapori diversi ad una certa distanza dalla sorgente (usati pricipalmente negli acceleratori);
	\item Esperimenti di scomparsa: dato il flusso alla sorgente di neutrini di un dato sapore, se ne misura il flusso a distanza L dalla sorgente. Se i neutrini oscillano il flusso sarà minore (neutrini solari e atmosferici);
	\item Short base-line experiment: esperimenti in cui il rivelatore è posto a piccola distanza dalla sorgente (10-1000 m);
	\item Long base-line experiment: esperimenti in cui il rivelatore è posto a grande distanza dalla sorgete (100-1000 km) (è tipico degli esperimenti per neutrini da acceleratore);
\end{itemize}

%%%%questo non ho capito che significa penso (e spero) che lo spieghi meglio nella teoria 
\noindent Il fenomeno dell'oscillazione significa che c'è una sovrapposizione delle onde che rappresentano gli autostati di massa dell'hamiltoniana che non sono né \textit{e} né $\mu$ né $\tau$ ma sono neutrini di tipo 1 2 e 3. Il neutrino che noi vediamo nel mondo reale è la convoluzione di neutrini di tipo 1 2 e 3 di cui esistono delle masse. Non è possibile con gli esperimenti distinguere le masse dei 3 tipi neutrini, ma è possibile misurare la differenza delle masse ($\Delta m^2$). Queste sovrapposizioni costituiscono poi i vari sapori. Allo stato attuale on sappiamo quale tipo abbia massa maggiore e quindi parliamo di "gerarchia di massa", identificarla è molto importante per comprendere il fenomeno delle oscillazioni. Alcuni degli elementi della matrice non sono ancora noti. Gli esperimenti su neutrini da reattore e da acceleratore mirano a misurare con precisione questi elementi di matrice. \\

%SLIDE 74 intervalli di parametro di massa 
I risulati dei vari esperimenti vengono messi insieme in dei plot detti "plot di esclusione" che servono a capire se ci sono discrepanze e anomalie. Tutte le anomalie che abbiamo rivelato hanno sempre segnalato qualcosa che non conoscevamo e che è legato alla natura del neutrino. \\

\subsection{I neutrini da reattore}
I neutrini da reattore hanno energie tra 0 e 10 MeV e differiscono in base a: composizione del core, energia massima raggiungibile e spettro energetico. Nei decadimenti di fissione vengono emessi antineutrini elettronici. Il flusso di questi neutrini è: \[ \Phi(\bar{\nu}_e)\ (\text{cm}^{-2}\,\text{s}^{-1}) \approx 1.5 \times 10^{12} \cdot \frac{P\ (\text{MW})}{L^2\ (\text{m}^2)}.\] 
Vengono prodotti in media 6 antineutrini per fissione. I neutrini vengono rivelati mediante decadimento $\beta$ inverso: vengono prodotti antineutrini elettronici, a valle (come in JUNO) pongo una sfera piena di scintillatore, i neutrini attraversandolo fanno scattering e le particelle diffuse producono luce di scintillazione o Cherenkov che viene rivelata. Vengono rivelati anche $\gamma$ prodotti per termalizzazione, cattura neutronica, o effetto Compton; questi ovviamente costituiscono un fondo. \\
L'esperimento CHOOZ è costituito da un rivelatore a 1 km dall’omonimo reattore nella regione delle Ardenne in Francia a 100 m sotto terra in una vecchia galleria, rivela antineutrini elettronici prodotti da reattore che interagiscono con 300 l di scintillatore liquido. Il fascio è estremamente puro ed è costituito al 100\% da antineutrini elettronici. \\

%%SLIDE 77 PLOT DI ESCLUSIONE: Delta m2 in funzione dell'altro parametro. Le curve rappresentano il limite tra la "regione permessa" e la "regione proibita", in questo modo identifico le regioni nello spazio dei parametri in cui è possibile che l'oscillazione avvenga. 

L'esperimento KamLAND (Kamioca Liquid-scintillator Anti-Neutrino Detector) ha dato la prima prova dell'oscillazione di neutrini da reattore e rivela neutrini di energia intermedia ($\sim 4$ MeV) prodotti da 53 reattori posizionati in Giappone e in Corea.\\
Questi due esperimenti hanno permesso di ottenere valori molto precisi di $\Delta m^2$, con sensibilità pari a: $\sim 10^-3 \,\text{(eV)}^2$ per CHOOZ e $\sim 10^-5 \,\text{(eV)}^2$ per KamLAND.


\subsection{Neutrini da acceleratore}
Negli anni 2000 l'Europa ha concentrato le risorse nel progetto CNGS (CERN Neutrino beam to Gran Sasso). Si utilizza un fascio di protoni prodotto dal SPS (Super Proton Sincroton) del CERN, sparati su un sottile bersaglio di grafite, e fatti decadere in volo in particelle secondarie (K e $\pi$); queste vengono focalizzate in un fascio mediante magneti deflettori e fatte scorrere in un tunnel di 1 km dove si formano i neutrini che viaggeranno fino al Gran Sasso (coprendo una distanza di 732 km). \\%se vi interessava rossella ha lavorato su questo come tesi di laurea
Un parametro molto importante è la probabilità che un neutrino di un certo sapore si trasformi in un altro, questa probabilità è proporzionale agli angoli di mescolamento, alla base e all'energia.\\
L'esperimento ICARUS (Imaging Cosmic and Rare Underground SIgnals) è un erede della camera a bolle. È una camera che funziona ad Argon liquido. Al passaggio delle particelle si formano delle coppie elettrone-ione e dei fotoni, i fotoni emessi danno il segnale di start e poi dei piani polatrizzati raccolgono gli elettroni. Questo apparato funziona quindi come bersaglio e rivelatore allo stesso tempo. %in realtà nemmeno lo ha detto ho caopiato la SLIDE 86



\end{document}  