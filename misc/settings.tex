%=================%
%   PAGE LAYOUT   %
%=================%

\settrims{0pt}{0pt}
\settypeblocksize{*}{1.1\lxvchars}{*}
\setbinding{1.5cm}
\setlrmargins{*}{*}{4}
\setulmarginsandblock{1in}{*}{2}
\checkandfixthelayout



%=========================%
%   HEADER & PAGESTYLES   %
%=========================%
\sideparmargin{outer}
\setmarginnotes{0.1\foremargin}{0.7\foremargin}{\onelineskip}
\aliaspagestyle{chapter}{empty}             % default \pagestyle{chapter} = \pagesstyle{plain} mette il numero al centro in basso.

\setlength{\headwidth}{\textwidth}
    \addtolength{\headwidth}{\marginparsep}
    \addtolength{\headwidth}{\marginparwidth}

\makepagestyle{own}
    \makerunningwidth{own}{\headwidth}
    \makeheadrule{own}{\headwidth}{\normalrulethickness}
    \makeheadposition{own}{flushright}{flushleft}{flushright}{flushleft}
    \makepsmarks{own}{%
        \nouppercaseheads
        \createmark{chapter}{both}{shownumber}{\chaptername\ }{.\ }
        \createmark{section}{right}{shownumber}{}{.\ }
        \createplainmark{toc}{both}{\contentsname}
        \createplainmark{lof}{both}{\listfigurename}
        \createplainmark{lot}{both}{\listtablename}
        \createplainmark{bib}{both}{\bibname}
        \createplainmark{index}{both}{\indexname}
        \createplainmark{glossary}{both}{\glossaryname}    
    }
    \makeevenhead{own}{\bfseries\thepage}{}{\bfseries\leftmark}
    \makeoddhead{own}{\bfseries\rightmark}{}{\bfseries \thepage}
    \makeevenfoot{own}{}{}{\footnotesize \theauthor}
    \makeoddfoot{own}{\footnotesize \theauthor}{}{}


%==============%
%   INITIALS   %
%==============%

\input Rothdn.fd
\newcommand*\initfamily{\usefont{U}{Rothdn}{xl}{n}}
\DeclareFontFamily{U}{yinit}{}
\DeclareFontShape{U}{yinit}{m}{n}{<-> yinit}{}
\newcommand{\initcolor}{purple}



%===============%
%   NUMBERING   %
%===============%

\numberwithin{equation}{chapter}    % Aggiunge il numero del capitolo all'equazione
\setsecnumdepth{subsection}         % numera le subsection (1.1.1)

\hypersetup{                    % ======================================
    colorlinks=true,            % template preso da internet
    linkcolor=black,            % molto sobrio, le cose cliccabili
    filecolor=black,            % si evidenziano quando passi il
    urlcolor=blue,              % mouse ma i link che vanno a capo
    pdfpagemode=FullScreen,     % si evidenziano solo sul rigo del mouse.
    citecolor=black,            % forse nel pdf finito non si colorano.
}



%==================%
%   BIBLIOGRAPHY   %
%==================%

\addbibresource{misc/references.bib}