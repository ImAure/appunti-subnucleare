%=================%
%   PAGE LAYOUT   %
%=================%

\settrims{0pt}{0pt}
\settypeblocksize{*}{1.1\lxvchars}{*}
\setbinding{1.5cm}
\setlrmargins{*}{*}{4}
\setulmarginsandblock{1in}{*}{2}
\checkandfixthelayout



%=========================%
%   HEADER & PAGESTYLES   %
%=========================%

\sideparmargin{outer}
\setmarginnotes{0.1\foremargin}{0.7\foremargin}{\onelineskip}
\aliaspagestyle{chapter}{empty}             % default \pagestyle{chapter} = \pagesstyle{plain} mette il numero al centro in basso.

\setlength{\headwidth}{\textwidth}
    \addtolength{\headwidth}{\marginparsep}
    \addtolength{\headwidth}{\marginparwidth}

\makepagestyle{own}
    \makerunningwidth{own}{\headwidth}
    \makeheadrule{own}{\headwidth}{\normalrulethickness}
    \makeheadposition{own}{flushright}{flushleft}{flushright}{flushleft}
    \makepsmarks{own}{%
        \nouppercaseheads
        \createmark{chapter}{both}{shownumber}{\chaptername\ }{.\ }
        \createmark{section}{right}{shownumber}{}{.\ }
        \createplainmark{toc}{both}{\contentsname}
        \createplainmark{lof}{both}{\listfigurename}
        \createplainmark{lot}{both}{\listtablename}
        \createplainmark{bib}{both}{\bibname}
        \createplainmark{index}{both}{\indexname}
        \createplainmark{glossary}{both}{\glossaryname}    
    }
    \makeevenhead{own}{\bfseries\thepage}{}{\bfseries\leftmark}
    \makeoddhead{own}{\bfseries\rightmark}{}{\bfseries \thepage}
    \makeevenfoot{own}{}{}{\footnotesize \theauthor}
    \makeoddfoot{own}{\footnotesize \theauthor}{}{}


%==============%
%   INITIALS   %
%==============%

\input Rothdn.fd
\newcommand*\initfamily{\usefont{U}{Rothdn}{xl}{n}}
\DeclareFontFamily{U}{yinit}{}
\DeclareFontShape{U}{yinit}{m}{n}{<-> yinit}{}
\newcommand{\initcolor}{purple}



%===============%
%   NUMBERING   %
%===============%

\numberwithin{equation}{chapter}    % Aggiunge il numero del capitolo all'equazione
\setsecnumdepth{subsection}         % numera le subsection (1.1.1)

\hypersetup{                    % ======================================
    colorlinks=true,            % template preso da internet
    linkcolor=black,            % molto sobrio, le cose cliccabili
    filecolor=black,            % si evidenziano quando passi il
    urlcolor=blue,              % mouse ma i link che vanno a capo
    pdfpagemode=FullScreen,     % si evidenziano solo sul rigo del mouse.
    citecolor=black,            % forse nel pdf finito non si colorano.
}



%==================%
%   BIBLIOGRAPHY   %
%==================%

\addbibresource{misc/references.bib}



%=======================%
%   TIKZ.NET SETTINGS   %
%=======================%

\usetikzlibrary{calc}
\usetikzlibrary{positioning}
\tikzset{>=latex} % for LaTeX arrow head

% UNSLANT GREEK LETTERS for particle symbols
% https://tex.stackexchange.com/questions/145926/upright-greek-font-fitting-to-computer-modern
% https://tex.stackexchange.com/questions/236915/adjust-custom-made-upright-greek-letters-when-used-in-subscripts
\usepackage{scalerel}
\newsavebox{\foobox}
\newcommand{\slantbox}[2][0]{\mbox{%
  \sbox{\foobox}{#2}%
  \hskip\wd\foobox
  \pdfsave
  \pdfsetmatrix{1 0 #1 1}%
  \llap{\usebox{\foobox}}%
  \pdfrestore
}}
\newcommand\unslant[2][-.25]{%
  %\mkern1.2mu%
  \ThisStyle{\slantbox[#1]{$\SavedStyle#2$}}%
  \mkern-2.2mu%
}
\newcommand\PGm{\unslant\mu} % muon
\newcommand\PGt{\unslant\tau} % tau
\newcommand\PGn[1]{\unslant\nu_{#1}\mkern-1.5mu} % neutrino

\makeatletter
\newcommand{\raisemath}[1]{\mathpalette{\raisem@th{#1}}}
\newcommand{\raisem@th}[3]{\raisebox{#1}{$#2#3$}}
\makeatother

% COLORS
\colorlet{mylightblue}{blue!60!cyan!80!black!15}
\colorlet{mypurple}{blue!50!red!70}
\colorlet{gaugecol}{red!90!black!70} % Wiki red
\colorlet{leptoncol}{green!80!black!70} % Wiki green
\colorlet{quarkcol}{blue!85!cyan!95!black!55} % Wiki purple
\colorlet{quarkred}{red!98!black!55} % quark red
\colorlet{quarkblue}{blue!85!cyan!98!black!55} % quark blue
\colorlet{quarkgreen}{green!95!black!55} % quark green
\colorlet{gluoncyan}{cyan!100!black!55} % gluon cyan
\colorlet{gluongreen}{green!75!blue!95!black!70} % gluon green
\colorlet{gluonyellow}{yellow!98!black!55} % gluon yellow
\colorlet{gluonorange}{orange!100!black!65} % gluon orange
\colorlet{gluonmagenta}{magenta!100!black!70} % gluon magenta
\colorlet{scalarcol}{yellow!70!orange!98!black}
\colorlet{tensorcol}{blue!50!red!70} % Wiki light blue
\colorlet{groupcol}{orange!15}

% STYLES
\tikzset{
  >=latex, % for LaTeX arrow head
  header/.style={black,midway,font=\bfseries,align=center,scale=0.6},
  proplabel/.style={black!70,scale=0.5}, % label of properties
  bflabel/.style={font=\bfseries,inner sep=0.5pt,rotate=90},
}

% LAYERS
\pgfdeclarelayer{back} % to draw on background
\pgfsetlayers{back,main} % set order

% PARTICLE macro
\def\pw{0.94} % width/height of particle box
\newcommand\opGen[2]{min(#1,(\setGen==0 || \setGen==#2) ? 1 : \d)} % calculate fermion opacity
\tikzset{
  x={(1.25,0)},y={(0,1.5)}, % scale x, y axes differently
  global scale/.style={scale=#1,every node/.style={scale=#1}},
  intgroup/.style={draw=#1!90!black!80,line width=0.5, % interaction groups
                   fill=#1,fill opacity=0.5},
  intgroup/.default=groupcol,
  pics/particle/.style n args={3}{ % particle boxes
    code={
      \tikzset{/tikz/pic opacity/.get=\OP}
      \begin{scope}[opacity=\OP]
      
      % COORDINATES
      \coordinate (-sw) at (-\pw/2,-\pw/2);
      \coordinate (-nw) at (-\pw/2, \pw/2);
      \coordinate (-se) at ( \pw/2,-\pw/2);
      \coordinate (-ne) at ( \pw/2, \pw/2);
      
      % DRAW BOX
      \ifnum\pgfkeysvalueof{/tikz/fill box}=1 % fill particle boxes with color
        \draw[#1,line width=1.1,rounded corners=3pt,shading angle=30,
              top color=#1!90!black!40,bottom color=#1!75!black!40]
          (-sw) rectangle (-ne);
      \else % do not fill particle boxes with color
        \fill[top color=#1,bottom color=#1!90!black,shading angle=30,
              rounded corners=3pt,even odd rule]
          (-0.48*\pw,-0.48*\pw) rectangle (0.48*\pw,0.48*\pw)
          [rounded corners=3.7pt] (-sw) rectangle (-ne);
      \fi
      
      % DRAW PARTICLES
      \ifnum\pgfkeysvalueof{/tikz/quark balls}>0 % draw QUARK as colored RGB balls
        \ifnum \pgfkeysvalueof{/tikz/quark balls}=1
          \def\quarklist{quarkgreen/35:4pt/0.9,quarkred/-35:3.7pt/0.9,quarkblue/0:0/1}
        \else
          \def\quarklist{gluonmagenta/35:4pt/0.9,gluoncyan/-35:3.7pt/0.9,gluonyellow/0:0/1}
        \fi
        \foreach \col/\shift/\scale in \quarklist{
          \fill[ball color=\col,scale=1,shift=(\shift),scale=\scale, % particle ball
                postaction={fill=\col!80,opacity=0.8*\OP,
                draw=\col!80!black!90,ultra thin}]
            (0,0.11) circle(9.5pt) coordinate(-p);
          }
      \else \ifnum\pgfkeysvalueof{/tikz/gluon balls}=1 % draw GLUON as colored RGB balls
        \foreach \col/\shift/\scale in {%
          gluonmagenta/-50:5pt/0.8,gluonyellow/0:5pt/0.78,gluoncyan/50:5pt/0.8,%
          gluongreen/100:6pt/0.8,quarkblue/200:5.8pt/0.8,gluonorange/150:5.8pt/0.76,%
          quarkgreen/250:4.5pt/0.8,quarkred/0:0/1%
        }{
          \fill[ball color=\col,scale=1,shift=(\shift),scale=\scale, % particle ball
                postaction={fill=\col!80,opacity=0.8*\OP,
                draw=\col!80!black!90,ultra thin}]
            (0,0.11) circle(9pt) coordinate(-p);
          }
      \else % draw one PARTICLE ball
        \draw[draw=none,ball color=#1,scale=1, % particle ball
              postaction={fill=#1!77,opacity=0.8*\OP,
              draw=#1!80!black!90,ultra thin}]
          (0,0.11) circle(\pgfkeysvalueof{/tikz/ball radius}) coordinate(-p);
      \fi \fi % close all \ifnum
      
      % DRAW TEXT
      \node[text=black,scale=1,shift=\pgfkeysvalueof{/tikz/symb shift}] % particle symbol
        at (-p) {\textbf{\boldmath{#2}}};
      \node[align=center,font=\bfseries, % particle name
            scale=0.7*\pgfkeysvalueof{/tikz/scale name}]
        at (0,-0.29) {\strut#3};
      
      \end{scope} % close opacity scope
    }
  },
  % DEFAULT SETTINGS of parameters:
  scale name/.initial=1, % scale for particle name
  symb shift/.initial={(0,0)}, % shift for particle symbol
  ball radius/.initial=10pt, % radius for particle ball
  quark balls/.initial=0, % draw quark as 3 RGB-colored balls
  gluon balls/.initial=0, % draw gluon as 8 RGB-colored balls
  fill box/.initial=0, % fill particle boxes
  pic opacity/.initial=1 % opacity of pictures
}

% HEADERS
\def\nfermioncols{3} % number of fermion columns, default = 3
\def\nbosoncols{2} % number of boson columns, default = 2
\def\headers{
  \fill[mylightblue,rounded corners=4pt] % FERMIONS
    (1-\pw/2,4.74) rectangle (3+\pw/2,5.1)
    node[midway,header] {%
      three generations of matter\\[0pt]
      (fermions)};
  \node[above=0pt,scale=0.75] at (1,4.5) {I};
  \node[above=0pt,scale=0.75] at (2,4.5) {II};
  \node[above=0pt,scale=0.75] at (3,4.5) {III};
  \ifnum\nfermioncols>3 % include antifermions
    \fill[mylightblue,rounded corners=4pt] % ANTIFERMIONS
      (4-\pw/2,4.74) rectangle (\nfermioncols+\pw/2,5.1)
      node[midway,header] {%
        three generations of antimatter\\[0pt]
        (antifermions)};
    \node[above=0pt,scale=0.75] at (4,4.5) {I};
    \node[above=0pt,scale=0.75] at (5,4.5) {II};
    \node[above=0pt,scale=0.75] at (6,4.5) {III};
  \fi
  \fill[mylightblue,rounded corners=4pt] % BOSONS
    (\nfermioncols+1-\pw/2,4.74) rectangle (\nfermioncols+\nbosoncols+\pw/2,5.1)
    node[midway,header] {%
      interactions / forces\\[0pt]
      (bosons)};
}
\def\headerMSSM{
  \fill[mylightblue,rounded corners=4pt] % SFERMIONS / BOSONS
    (1-\pw/2,4.74) rectangle (3+\pw/2,5.1)
    node[midway,header] {%
      superpartners of SM fermions\\[0pt]
      (sfermions, bosons)};
  \node[above=0pt,scale=0.75] at (1,4.5) {I};
  \node[above=0pt,scale=0.75] at (2,4.5) {II};
  \node[above=0pt,scale=0.75] at (3,4.5) {III};
  \fill[mylightblue,rounded corners=4pt] % BOSINOS / FERMIONS
    (\nfermioncols+1-0.52*\pw,4.74) rectangle (\nfermioncols+\nbosoncols+0.52*\pw,5.1)
    node[midway,header,scale=0.96,xscale=0.92] {%
      superpartners of SM bosons\\[0pt]
      (bosinos, fermions)};
}